\documentclass[10pt,landscape]{article}

\usepackage{multicol}
\usepackage{calc}
\usepackage{ifthen}
\usepackage[landscape]{geometry}

\usepackage{amsmath}
\usepackage{amssymb}
\usepackage{gauss}

%% taken from http://brunoj.wordpress.com/2009/10/08/latex-the-framed-minipage/
\newsavebox{\fmbox}
\newenvironment{fmpage}[1]
{\begin{lrbox}{\fmbox}\begin{minipage}{#1}}
{\end{minipage}\end{lrbox}\fbox{\usebox{\fmbox}}}

\usepackage{mathtools}
\makeatletter
 
\newcommand{\explain}[2]{\underset{\mathclap{\overset{\uparrow}{#2}}}{#1}}
\newcommand{\explainup}[2]{\overset{\mathclap{\underset{\downarrow}{#2}}}{#1}}
 
\makeatother

% To make this come out properly in landscape mode, do one of the following
% 1.
%  pdflatex latexsheet.tex
%
% 2.
%  latex latexsheet.tex
%  dvips -P pdf  -t landscape latexsheet.dvi
%  ps2pdf latexsheet.ps


% If you're reading this, be prepared for confusion.  Making this was
% a learning experience for me, and it shows.  Much of the placement
% was hacked in; if you make it better, let me know...


% 2008-04
% Changed page margin code to use the geometry package. Also added code for
% conditional page margins, depending on paper size. Thanks to Uwe Ziegenhagen
% for the suggestions.

% 2006-08
% Made changes based on suggestions from Gene Cooperman. <gene at ccs.neu.edu>


% To Do:
% \listoffigures \listoftables
% \setcounter{secnumdepth}{0}


% This sets page margins to .5 inch if using letter paper, and to 1cm
% if using A4 paper. (This probably isn't strictly necessary.)
% If using another size paper, use default 1cm margins.
\ifthenelse{\lengthtest { \paperwidth = 11in}}
	{ \geometry{top=.5in,left=.5in,right=.5in,bottom=.5in} }
	{\ifthenelse{ \lengthtest{ \paperwidth = 297mm}}
		{\geometry{top=1cm,left=1cm,right=1cm,bottom=1cm} }
		{\geometry{top=1cm,left=1cm,right=1cm,bottom=1cm} }
	}

% Turn off header and footer
\pagestyle{empty}
 

% Redefine section commands to use less space
\makeatletter
\renewcommand{\section}{\@startsection{section}{1}{0mm}%
                                {-1ex plus -.5ex minus -.2ex}%
                                {0.5ex plus .2ex}%x
                                {\normalfont\large\bfseries}}
\renewcommand{\subsection}{\@startsection{subsection}{2}{0mm}%
                                {-1explus -.5ex minus -.2ex}%
                                {0.5ex plus .2ex}%
                                {\normalfont\normalsize\bfseries}}
\renewcommand{\subsubsection}{\@startsection{subsubsection}{3}{0mm}%
                                {-1ex plus -.5ex minus -.2ex}%
                                {1ex plus .2ex}%
                                {\normalfont\small\bfseries}}
\makeatother

% Define BibTeX command
\def\BibTeX{{\rm B\kern-.05em{\sc i\kern-.025em b}\kern-.08em
    T\kern-.1667em\lower.7ex\hbox{E}\kern-.125emX}}

% Don't print section numbers
\setcounter{secnumdepth}{0}


\setlength{\parindent}{0pt}
\setlength{\parskip}{0pt plus 0.5ex}


% -----------------------------------------------------------------------

\begin{document}

\raggedright
\footnotesize
\begin{multicols}{3}


% multicol parameters
% These lengths are set only within the two main columns
%\setlength{\columnseprule}{0.25pt}
\setlength{\premulticols}{1pt}
\setlength{\postmulticols}{1pt}
\setlength{\multicolsep}{1pt}
\setlength{\columnsep}{2pt}

\begin{center}
     \normalsize{Linear Modeling Summary Sheet} \\
    \footnotesize{
    Compiled by: Shravan Vasishth (vasishth@uni-potsdam.de)\\
    Version dated: \today}
\end{center}

%\textbf{Background for this summary sheet}

%This is a summary for myself, for reviewing the material I learnt while doing an MSc-level course in statistics; your mileage may vary.
Note: Everything is in matrix form unless a lower case letter with a subscript (such as $x_i$) is used.


\section{Background}

\subsection{Some key distributional results}

\section{Basic facts}

\begin{equation}
y=X\beta+\epsilon
\end{equation}

\begin{tabular}{@{}ll@{}ll@{}}
$E(y) = X\beta = \mu$ &  & $E(\epsilon)=0$  \\
$Var(y) = \sigma^2 I_n $ & & $Var(\epsilon) = \sigma^2 I_n$ \\
%y+X\hat{\beta} + e$ \\
%%$E(y) = X\beta = \mu$ & Var\\
% $\epsilon \sim N_p (0,\sigma^2I_n)$  & No \verb!\part! divisions. \\
%\verb!article! & No \verb!\part! or \verb!\chapter! divisions. \\
%\verb!letter!  & Letter (?). \\
%\verb!slides!  & Large sans-serif font.
\end{tabular}

\begin{equation}
y = X\hat{\beta} + e
\end{equation}

\begin{tabular}{@{}ll@{}ll@{}}
Results for $\hat{\beta}$ & Results for $e$\\
$E(\hat{\beta}) = \beta$ & $E(e) = 0$\\
$Var(\hat{\beta}) = \sigma^2 (X^T X)^{-1} = \frac{\sigma^2}{S_{xx}}$ & $Var(e)=\sigma^2 M$  \\
$\hat{\beta} \sim N_p(\beta,\sigma^2 (X^T X)^{-1})$ & $Var(e_i)=\sigma^2 m_{ii} $\\ 
&  $E(e_i^2)= \sigma^2 m_{ii}$\\
$\hat{\beta} = (X^T X)^{-1} X^T y$, $X$ has full rank  & $E(\sum e_i^2) = \sigma^2 (n-p)$\\
\end{tabular}

\medskip
\textbf{Sum of Squares}:

\begin{equation}
S(\hat{\beta}) = \sum e_i^2 = e^T e = (y-X\hat{\beta})^T (y-X\hat{\beta})  = y^T y - y^T X \hat{\beta} = S_r
\end{equation}

\textbf{Estimation of error variance: $e=My$}

\begin{equation}
e = y - X\hat{\beta} = y - X (X^T X)^{-1} X^T y = My
\end{equation}

\noindent
where

\begin{equation}
M = I_n -  X (X^T X)^{-1} X^T \quad \hbox{M is symmetric, idempotent } n\times n
\end{equation}

Note that $MX=0$, which means that 

\begin{equation}
E(e)=E(My) = ME(y)= MX\beta = 0
\end{equation}

Also, $Var(e) = Var(My) = M Var(y) M^T = \sigma^2 I_n M$.

\medskip
\textbf{Important properties of M}:

\begin{itemize}
\item $M$ is singular because every idempotent matrix except $I_n$ is singular.
\item $trace(M)=rank(M)=n-p$.
\end{itemize}

\medskip


\textbf{Residual mean square}:
\begin{equation}
\hat{\sigma}^2 = \frac{\sum e_i^ 2}{n-p} \quad E(\hat{\sigma}^2)=\sigma^2
\end{equation}

The square root of $\hat{\sigma}^2$, $\hat{\sigma}$ is the \textbf{residual standard error}.

\subsection{Some short-cuts for hand-calculations}

\begin{tabular}{@{}ll@{}}
$S_{xx} = \sum (x_i - \bar{x})^2$  & $= \sum x_i^2 - n\bar{x}^2$ \\
$S_{yy} = \sum (y_i - \bar{y})^2$ & $= \sum y_i^2 - n\bar{y}^2$\\ 
$S_{xy} = \sum (x_i -\bar{x})(y_i -\bar{y})$ & = $\sum x_i y_i - n\bar{x}\bar{y}$\\
\end{tabular}

\begin{equation}
\hat{\beta} = (X^T X)^{-1} X^T y = 
\begin{pmatrix} 
\bar{y} - \bar{x} \frac{S_{xy}}{S_{xx}}\\
\frac{S_{xy}}{S_{xx}}
\end{pmatrix}
\end{equation}

See \cite[25]{DraperSmith} for a full exposition.

\subsection{Gauss-Markov conditions}

This imposes distributional assumptions on $\epsilon = y - X \beta$.

$E(\epsilon)=0$ and $Var(\epsilon)=\sigma^2 I_n$,

\subsection{Gauss-Markov theorem}

Let $a$ be any $p \times 1$ vector and suppose that $X$ has rank $p$. Of all estimators of $\theta = a^T \beta$ that are unbiased and linear functions of $y$, the estimator $\hat{\theta} = a^T \hat{\beta}$ has minimum variance. Note that $\theta$ is a scalar.

Note: no normality assumption required! But if $\epsilon \sim N(0,\sigma^2)$, $\hat{\beta}$ have smaller variances than any other estimators.

\subsection{Coefficient of determination}

\begin{tabular}{@{}ll@{}}
$S_{TOTAL} = (y-\bar{y})^T(y-\bar{y})$  $= y^T y - n\bar{y}^2$ & \\
$S_{REG} = (X\hat{\beta}-\bar{y})^T (X\hat{\beta}-\bar{y})$ & \\
$S_r = \sum e_i^2 = (y-X\hat{\beta})^T (y-X\hat{\beta})$ & \\
\end{tabular}

\begin{equation}
S_{TOTAL} = S_{REG}+ S_r
\end{equation}

\begin{equation}
R^2 = \frac{S_{TOTAL}-S_r}{S_{TOTAL}} = \frac{S_{REG}}{S_{TOTAL}}
\end{equation}

For $y = 1_n \beta_0 + \epsilon$, then $R^2 = \frac{S_{REG}}{S_{TOTAL}} = 0$ because $X\hat{\beta} = \bar{y}$. So $S_{REG} = (X\hat{\beta} - \bar{y})^T (X\hat{\beta} - \bar{y}) = 0$.

In simple linear regression, $R^2 = r^2$.  $R^2$ is a generalization of $r^2$.

Adjusted $R^2= R_{Adj}^2$.  $R_{Adj}^2= 1-\frac{S_r/(n-p)}{S_{TOTAL}/(n-1)}$. 

$R^2$ increases with increasing numbers of explanatory variables, therefore $R_{Adj}^2$ is better. 


\section{Hypothesis testing}

\subsection{Some theoretical background}

\textbf{Multivariate normal}:

Let $X^T = < X_1,\dots,X_p>$, where $X_i$ are univariate random variables.

X has a multivariate normal distribution if and only if every component of $X$ has a univariate normal distribution.


\textbf{Linear transformations}:

Let $A, b$ be constants. Then, $Ax + b\sim N_q (A\mu + b, A\Sigma A^T)$.

\textbf{Standardization}:

Note that $\Sigma$ is positive definite (it's a variance covariance matrix), so $\Sigma = CC^T$. 
$C$ is like a square root (not necessarily unique).
 
It follows ``immediately'' that 

\begin{equation}
C^{-1} (X-\mu) \sim N_p (0_p, I_p)
\end{equation}

If $\Sigma$ is a diagonal matrix, then $X_1,\dots,X_n$ are independent and uncorrelated.

\textbf{Quadratic forms}:

Recall distributional result: If we have $n$ independent standard normal random variables, their sum of squares is $\chi_n^2$.

Lt $z = C^{-1} (X-\mu)$, and $\Sigma=CC^T$. The sum of squares $z^T z$ is:

\begin{equation}
\begin{split}
z^T z & = [C^{-1} (X-\mu)]^T [C^{-1} (X-\mu)]\\
& = (X-\mu)^T [C^{-1}]^T [C^{-1}](X-\mu) \quad \dots (AB)^T=B^T A^T\\
\end{split}
\end{equation} 

Note that $ [C^{-1}]^T =  [C^{T}]^{-1}$. Therefore, 

\begin{equation}
\begin{split}
[C^{-1}]^T [C^{-1}] & = [C^T]^{-1} [C^{-1}]\\
& = (C^T C)^{-1}\\
& = (C C^T)^{-1}\\
& = \Sigma^{-1}\\
\end{split}
\end{equation} 

Therefore: $z^T z = (X-\mu)^T  \Sigma^{-1} (X-\mu)\sim \chi_p^2$, where $p$ is the number of parameters.

\textbf{Quadratic expressions involving idempotent matrices}

Given a matrix $K$ that is idempotent, symmetric. Then:

\begin{equation}
x^T K x = x^T K^2 x = x^T K^T K x
\end{equation}

Let $x\sim N_n(\mu,\sigma^2 I_n)$, and let $K$ be a symmetric, idempotent $n \times n$ matrix such that $K\mu=0$. Let $r$ be the rank or trace of $K$. Then we have the \textbf{sum of squares property}:

\begin{equation}
x^T K x \sim \sigma^2 \chi_r^2
\end{equation}

The above generalizes the fact that if we have $n$ independent standard normal random variables, their sum of squares is $\chi_n^2$.

Two points about the sum of squares property:
\begin{itemize}
\item
Recall that the expectation of a chi-squared random variable is its degrees of freedom. It follows that:

\begin{equation}
E(x^T K x) =  \sigma^2 r 
\end{equation}

If $K\mu\neq 0$, $E(x^T K x) =  \sigma^2 r+\mu^T K\mu$. 

\item If $K$ is idempotent, so is $I-K$. This allows us to split $x^T x$ into two components sums of squares:

\begin{equation}
x^T x = x^T K x+x^T (I-K) x
\end{equation}
\end{itemize}

\textbf{Partition sum of squares}: 

 Let $K_1, K_2,\dots, K_q$ be symmetric idempotent $n \times n$ matrices such that
 $\sum K_i= I_n$ and $K_iK_j =0$, for all $i\neq j $. Let $x\sim N_n(\mu, \sigma^2)$.
 Then we have the following partitioning into independent sums of squares:
 
  \begin{equation}
x^T x = \sum x^T K_i x
\end{equation}

If $K_i \mu = 0$, then $ x^T K_i x\sim \sigma^2 \chi_{r_i}^2$, where $r_i$ is the rank of $K_i$.

\subsection{Confidence intervals for $\hat{\beta}$}

Note that $\hat{\beta} \sim N_p (\beta,\sigma^2 (X^T X)^{-1})$, and that 
$\frac{\hat{\sigma}^2}{\sigma^2} \sim  \frac{\chi^2_{n-p}}{n-p}$.

From distributional theory we know that $T=\frac{X}{\sqrt{Y/v}}$, when $X\sim N(0,1)$ and $Y\sim \chi^2_{v}$. 

Let 
 $x_i$ be a column vector containing the values of the explanatory/regressor variables for a new observation $i$. Then if we define:

\begin{equation}
X=\frac{x_i^T \hat{\beta} - x_i^T \beta}{\sqrt{\sigma^2 x_i^T (X^T X)^{-1}x_i}} \sim N(0,1)
\end{equation}

\noindent
and 

\begin{equation}
Y=\frac{\hat{\sigma}^2}{\sigma^2} \sim  \frac{\chi^2_{n-p}}{n-p}
\end{equation}


It follows that  $T=\frac{X}{\sqrt{Y/v}}$:

\begin{equation}
T=  \frac{x_i^T \hat{\beta} - x_i^T \beta}{\sqrt{\hat{\sigma}^2 x_i^T (X^T X)^{-1}x_i}} = 
\frac{  \frac{x_i^T \hat{\beta} - x_i^T \beta}{\sqrt{\sigma^2 x_i^T (X^T X)^{-1}x_i}}}{\sqrt{\frac{\hat{\sigma}^2}{\sigma^2}}}
 \sim t_{n-p}
\end{equation}

I.e., a 95\% CI:

\begin{equation}
x_i^T \hat{\beta} \pm t_{n-p,1-\alpha/2}\sqrt{\hat{\sigma}^2 x_i^T(X^T X)^{-1}x_i}
\end{equation}

Cf.\ a prediction interval:

\begin{equation}
x_i^T \hat{\beta} \pm t_{n-p,1-\alpha/2}\sqrt{\hat{\sigma}^2 (1+x_i^T(X^T X)^{-1}x_i)}
\end{equation}

Note that a prediction interval will be wider about the edges.

\subsection{Distributions of estimators and residuals}

Covar$(\hat{\beta},e)=0$: 

         Var$\begin{pmatrix}
	 \hat{\beta} \\
	e \\
	\end{pmatrix}
	= 
	\begin{pmatrix}
	 Var(\hat{\beta}) & 0 \\
	 0 & Var(e) \\
	\end{pmatrix}
	= 
	\begin{pmatrix}
	 \sigma^2 (X^T X)^{-1} & 0 \\
	 0 & \sigma^2 M \\
	\end{pmatrix}
	$.
	
\textbf{Confidence intervals for components of $\beta$}	


Let $G=(X^T X)^{-1}$, and $g_{ii}$ the $i$-th diagonal element. 

\begin{equation}
\hat{\beta}_i \sim N(\beta_i, \sigma^2 g_{ii})
\end{equation}

Since $\hat{\beta}$ and $S_r$ are independent, we have:

\begin{equation}
\frac{\hat{\beta}_i - \beta_i}{\hat{\sigma}\sqrt{g_{ii}}} \sim t_{n-p}
\end{equation}

The 95\% CI:


\begin{equation}
\hat{\beta}_i \pm t_{n-p,(1-\alpha)/2} \hat{\sigma} \sqrt{g_{ii}} 
\end{equation}

\subsection{Maximum likelihood estimators}

to-do

\subsection{Hypothesis testing}

A general format for specifying null hypotheses: $H_0: C\beta = c$, where $C$ is a $q\times p$ matrix and $c$ is a $q\times 1$ vector of known constants. The matrix $C$ effectively asserts specific values for $q$ linear functions of $\beta$. In other words, it asserts $q$ null hypotheses stated in terms of (components of) the parameter vector $\beta$.

 E.g., given:

\begin{equation}
y_i = \beta_0 + \beta_1 x_i + \beta_2 x_i^2+\epsilon_i
\end{equation}

\noindent
we can test $H_0: \beta_1=1, \beta_2=2$ by setting 

$C=\begin{pmatrix} 
0 & 1 & 0\\
0 & 0 & 1\\
\end{pmatrix}$
and $c=\begin{pmatrix} 
1\\
2\\
\end{pmatrix}$.

The alternative is usually the negation of the null, i.e., $H_1: C\beta\neq c$, which means that at least one of the $q$ linear functions does not take its hypothesized value. 

\textbf{Constructing a test}:

\begin{equation}
C\hat{\beta} \sim N_q (c,\sigma^2 C (X^T X)^{-1} C^T)
\end{equation}

So, if $H_0$ is true, by sum of squares property:

\begin{equation}
(C\hat{\beta} - c)^T C (X^T X)^{-1} C^T (C\hat{\beta} - c) \sim \sigma^2 \chi_q^2
\end{equation}

In other words:

\begin{equation}
\frac{(C\hat{\beta} - c)^T C (X^T X)^{-1} C^T (C\hat{\beta} - c)}{ \sigma^2} \sim \chi_q^2
\end{equation}

Note that $\hat{\beta}$ is independent of $\hat{\sigma}^2$, and recall that

\begin{equation}
\frac{\hat{\sigma}^2 }{\sigma^2} \sim \frac{\chi_{n-p}^2}{n-p}  \Leftrightarrow 
\frac{\hat{\sigma}^2 (n-p)}{\sigma^2} \sim \chi_{n-p}^2
\end{equation} 

Recall distributional result: if $X\sim \chi_v^2, Y\sim \chi_w^2$ and $X,Y$ independent then $\frac{X/v}{Y/w}\sim F,v,w$.  


It follows that if $H_0$ is true,  and setting
$X=\frac{(C\hat{\beta} - c)^T C (X^T X)^{-1} C^T (C\hat{\beta} - c)}{ \sigma^2}$,
$Y=\frac{\hat{\sigma}^2 (n-p)}{\sigma^2}$, and setting the degrees of freedom to $v=q$ and $w=n-p$:

\begin{equation}
\frac{X/v}{Y/w}=
\frac{\frac{(C\hat{\beta} - c)^T C (X^T X)^{-1} C^T (C\hat{\beta} - c)}{ \sigma^2}/q}{\frac{\hat{\sigma}^2 (n-p)}{\sigma^2}/(n-p)}
\end{equation}

Simplifying:

\begin{equation}
\frac{(C\hat{\beta} - c)^T C (X^T X)^{-1} C^T (C\hat{\beta} - c)}{q\hat{\sigma}^2} \sim F_{q,n-p}
\end{equation}

%The above test is the \textbf{generalized likelihood ratio test}.

This is a \textbf{one-sided test} even though the original alternative was two-sided.

\textbf{Special cases of hypothesis tests}:

When $q$ is 1, we have only one hypothesis to test, the $i$-th element of $\beta$. Given:

\begin{equation}
y_i = \beta_0 + \beta_1 x_i + \beta_2 x_i^2+\epsilon_i
\end{equation}

\noindent
we can test $H_0: \beta_1=0$ by setting 

$C=\begin{pmatrix} 
0 & 1 & 0\\
\end{pmatrix}$
and $c=0$.


Using the fact that $X\sim t(v)\Leftrightarrow X^2 \sim F(1,v)$, we have


\begin{equation}
\frac{\hat{\beta}_i - c_i}{\hat{\sigma}\sqrt{g_{ii}}} \sim t_{n-p}
\end{equation}
 
 \subsection{Sum of squares}

\begin{fmpage}{\linewidth}

Recall:

If $K$ is idempotent, so is $I-K$. This allows us to split $x^T x$ into two components sums of squares:

\begin{equation}
x^T x = x^T K x+x^T (I-K) x
\end{equation}

 Let $K_1, K_2,\dots, K_q$ be symmetric idempotent $n \times n$ matrices such that
 $\sum K_i= I_n$ and $K_iK_j =0$, for all $i\neq j $. Let $x\sim N_n(\mu, \sigma^2)$.
 Then we have the following partitioning into independent sums of squares:
 
  \begin{equation}
x^T x = \sum x^T K_i x
\end{equation}

If $K_i \mu = 0$, then $ x^T K_i x\sim \sigma^2 \chi_{r_i}^2$, where $r_i$ is the rank of $K_i$.
\end{fmpage}

We can use the sum of squares property just in case $K$ is idempotent, and $K\mu =0$ . Below, $K=M$ and $\mu=E(y)=X\beta$.

Consider the sum of squares partition:

\begin{equation}
y^T y = \explain{\underline{y^T M y}}{S_r= e^T e} + \explain{\underline{y^T (I-M) y}}{\hat{\beta}^T (X^T X)\hat{\beta}}
\end{equation}

Note that the preconditions for sums of squares partitioning are satisfied:
\begin{enumerate}
\item $M$ is idempotent  (and symmetric), rank=trace=$n-p$.
\item $I-M$ is idempotent (and symmetric), rank=trace=$p$.
\item $ME(y) = 0$ because $ME(y)=MX\beta$ and $MX=0$.
\end{enumerate}

We can therefore partition the sum of squares into two independent sums of squares:

\begin{equation}
y^T y = \explain{\underline{y^T M y}}{e^T e \sim \sigma^2 \chi_{n-p}^2} \hbox{~~~~~~~~}+\hbox{~~~~~~~~} 
\explain{\underline{y^T (I-M) y}}{ \sim \sigma^2 \chi_p^2 \newline \hbox{ iff } X\beta=0, i.e., \beta=0}
\end{equation}

So, iff we have $H_0: \beta=0$, we can partition sum of squares as above. Saying that $\beta=0$ is equivalent to saying that $X$ has rank $p$ and $X\beta=0$.

\subsection{Testing the effect of a subset of regressor variables}

Let:

\begin{equation}
C= (0_{p-q} I_q) \quad c=0, \hbox{ and } \beta=\begin{pmatrix} \beta_1\\ \beta_2 \end{pmatrix}
\end{equation}


Here, $\beta_{1,2}$ are vectors (sub-vectors?), not components of the $\beta$ vector.
Then, $C\times \beta = \beta_2$ and $H_0: \beta_2=0$. Note that order of elements in $\beta$ is arbitrary; i.e., any subset of $\beta$ can be tested.

Since  $C\times \beta = \beta_2$ and $c=0$, we can construct a sum of squares:

\begin{equation}
(C\hat{\beta} - c)^T C (X^T X)^{-1} C^T (C\hat{\beta} - c) \sim \sigma^2 \chi_q^2
\end{equation}

This becomes (since $C\beta=\hat{\beta}_2$):

\begin{equation}
\hat{\beta}_2^T C (X^T X)^{-1} C^T \hat{\beta}_2 \sim \sigma^2 \chi_q^2
\end{equation}

We can rewrite this as: $\hat{\beta}_2^T G_{qq} \hat{\beta}_2$, where $G_{qq}= C (X^T X)^{-1} C^T$ ($G_{qq}$ should not be confused with $g_{ii}$) is a $q\times q$ submatrix of $G=(X^T X)^{-1}$. 

Note that $\hat{\beta}$ is independent of $\hat{\sigma}^2$, and 
recall that  $\frac{\hat{\sigma}^2 (n-p)}{\sigma^2} \sim \chi_{n-p}^2$. We can now construct the F-test as before:

\begin{equation}
\frac{\hat{\beta}_2^T C (X^T X)^{-1} C^T \hat{\beta}_2}{q\hat{\sigma}^2} = 
\frac{\hat{\beta}_2^T G \hat{\beta}_2}{q\hat{\sigma}^2}
 \sim F_{q,n-p}
\end{equation}


\textbf{Sums of squares}:

We can construct three idempotent matrices:

\begin{itemize}
\item
$M = I_n - X(X^T X)^{-1} X^T$
\item
$M_1 =  X(X^T X)^{-1} X^T -   [X(X^T X)^{-1} C^T]  [\explain{\underline{C (X^T X)^{-1} C^T}}{G}]^{-1}  
[C(X^T X)^{-1} X^T]$

(that is: $M_1 =  X(X^T X)^{-1} X^T - M_2$)

\item $M_2 = [X(X^T X)^{-1} C^T]  [\explain{\underline{C (X^T X)^{-1} C^T}}{G}]^{-1}  
[C(X^T X)^{-1} X^T]$
\end{itemize}

Note that $M+M_1+M_2=I_n$ and $MM_1=MM_2=M_1M_2=0$. I.e., sum of squares partition property applies. We have three independent sums of squares:

\begin{enumerate}
\item $S_r = y^T M y$
\item $S_1 = y^T M_1 y = \hat{\beta}^T X^T X \hat{\beta}-  \hat{\beta}_2^T G_{qq}^{-1} \hat{\beta}_2$
\item $S_2 = y^T M_2 y =  \hat{\beta}_2^T G_{qq}^{-1} \hat{\beta}_2$
\end{enumerate}

So: $y^T y = S_r + S_1 + S_2$.  Then:

\begin{itemize}
\item It is unconditionally true that $S_r \sim \sigma^2 \chi^2_{n-p}$.
\item If $H_0: \beta=0$ is true, then $E(\hat{\beta}_2) = \beta_2 = 0$. It follows from the sum of squares property that $S_2 \sim \sigma^2 \chi_q^2$.  
\item Regarding $S_1$: 
We can prove that $M_1 = X_1 (X_1^T X_1)^{-1}X_1^T$, where $X_1$ contains the first $p-q$ columns of $X$. It follows that:

$S_1 = y^T M_1 y =y^T X_1 (X_1^T X_1)^{-1}X_1^T  y$

Note that $X_1 (X_1^T X_1)^{-1}X_1^T$ is idempotent. If $\beta=0$, i.e., if $E(y) =X\beta = 0$, we can use the  sum of squares property and conclude that

$S_1 \sim \sigma^2 \chi_{p-q}^2$

The degrees of freedom are $p-q$ because the rank=trace of  $X_1 (X_1^T X_1)^{-1}X_1^T$ is $n-p$.

\textbf{Thus, $S_1$ is testing $\beta_1=0$ but under the assumption that $\beta_2=0$}.

\end{itemize}

\textbf{Analysis of variance}

%\begin{table}[htdp]
%\caption{default}
%\begin{center}
\begin{tabular}{|l|c|c|c|c|}
\hline
Sources & SS & df & MS & MS ratio\\
 of variation & & & & \\
\hline
Due to $X_1$  & $S_1$ & $p-q$ & $S_1/(p-q)$ & $F_1$ \\
if $\beta_2=0$   d& & & & $F_{p-q,n-p}$\\
\hline
Due to $X_2$ & $S_2$ & $q$ & $S_2/q$ & $F_2$\\
& & & &  $F_{q,n-p}$\\
\hline
Residuals   & $S_r$ & $n-p$ & $\hat{\sigma}^2$ & \\
\hline
Total           & $y^T y$  & n &  &\\
\hline
\end{tabular}
%\end{center}
%\label{default}
%\end{table}%

Note:

\begin{enumerate}
\item 
The ANOVA tests are \textbf{performed in order}:  First we test $H_0: \beta_2=0$. Then, if this test does not reject the null, we test $H_0: \beta_1 = 0$ \textbf{on the assumption (which may or may not be true)} that $\beta_2=0$. 
\item What happens if we reject the first hypothesis?
\end{enumerate}

\textbf{The null or minimal model (constant term)}

We can set $C=I_p$ and $c=0$. This tests whether all coefficients are zero. But this states that $E(y)=0$, whereas it should have a non-zero value (e.g., reading times).  We include the constant term to accommodate this desire to have $E(y)=\mu=\neq 0$. In matrix format: let $\beta$ be the parameter vector; then, $\beta_1=\mu$ is the first, constant, term, and the rest of the parameters are the vector $\beta_2$ ($p-1\times 1$).
The first column of $X$ will be $X_1=1_n$.

\begin{enumerate}
\item
$S_1=y^T (X_1^T X_1)^{-1} X_1^T y = (\sum y)^2/n = n\bar{y}^2$
\item
$S_r = y^Ty - \hat{\beta}^T X^T X\hat{\beta}$
\item
$S_2 = y^T y -S_1 - S_r = \hat{\beta}^T X^T X\hat{\beta}-n\bar{y}^2$
\end{enumerate}

It is normal to omit the row in the ANOVA table corresponding to the constant term.

\medskip
\textbf{Testing whether all predictors (besides the constant term) are zero}

To test whether $p$ predictor variables have any effect on $y$,we set $q=p-1$, and our anova table looks like this:

\begin{tabular}{|l|l|l|l|l|}
\hline
Sources & SS & df & MS & MS \\
of variation  & & & & ratio \\
\hline
%Due to $X_1$ if $\beta_2=0$ & $S_1$ & $p-q$ & $S_1/(p-q)$ & ($F_1$) $  F_{p-q,n-p}$\\
%\hline
Due & $S_2$ & $p-1$ & $\frac{S_2}{(p-1)}$ & $F_2$\\
 to regressors & & & &  $F_{p-1,n-p}$\\
\hline
Residuals   & $S_r$ & $n-p$ & $\hat{\sigma}^2$ & \\
\hline
Total        & $S_{yy}=$  & n-1 &  &\\
 (\textbf{adjusted})    & $(y-\bar{y})^T(y-\bar{y})$ & & & \\
& $=y^T y - n\bar{y}^2$ & & & \\
\hline
\end{tabular}

Note that $S_{yy}=\sum (y_i - \bar{y})^2$ is the residual sum of squares that we get after fitting the constant $\hat{\mu}=\bar{y}$.

\medskip
\textbf{Testing a subset of predictors $\beta_2$}

\begin{tabular}{|l|l|l|l|l|}
\hline
Sources & SS & df & MS & MS \\
of variation  & & & & ratio \\
\hline
Due to $X_1$   & $S_1$ & $p-q-1$ & $\frac{S_1}{(p-q-1)}$ & ($F_1$) \\
if $\beta_2=0$ & & & & $F_{p-q-1,n-p}$\\
(test of $\beta_1$) & & & & \\
\hline
Due & $S_2$ & $q$ & $\frac{S_2}{q}$ & $F_2$\\
 to $X_2$ & & & &  $F_{q,n-p}$\\
(test of $\beta_2$) & & & & \\
\hline
Residuals   & $S_r$ & $n-p$ & $\hat{\sigma}^2$ & \\
\hline
Total        & $S_{yy}$  & n-1 &  &\\
\hline
\end{tabular}
Note: the lecture notes have total SS as $y^T y$ but I think that's a typo.

%Used at the very beginning of a document:
%\verb!\documentclass{!\textit{class}\verb!}!.  Use
%\verb!\begin{document}! to start contents and \verb!\end{document}! to
%end the document.




\bibliographystyle{plain}
\bibliography{/Users/shravanvasishth/Dropbox/Bibliography/bibcleaned}

Cheat sheet template taken from Winston Chang: http://www.stdout.org/$\sim$winston/latex/



\end{multicols}
\end{document}


\subsection{Common \texttt{documentclass} options}
\newlength{\MyLen}
\settowidth{\MyLen}{\texttt{letterpaper}/\texttt{a4paper} \ }
\begin{tabular}{@{}p{\the\MyLen}%
                @{}p{\linewidth-\the\MyLen}@{}}
\texttt{10pt}/\texttt{11pt}/\texttt{12pt} & Font size. \\
\texttt{letterpaper}/\texttt{a4paper} & Paper size. \\
\texttt{twocolumn} & Use two columns. \\
\texttt{twoside}   & Set margins for two-sided. \\
\texttt{landscape} & Landscape orientation.  Must use
                     \texttt{dvips -t landscape}. \\
\texttt{draft}     & Double-space lines.
\end{tabular}

Usage:
\verb!\documentclass[!\textit{opt,opt}\verb!]{!\textit{class}\verb!}!.


\subsection{Packages}
\settowidth{\MyLen}{\texttt{multicol} }
\begin{tabular}{@{}p{\the\MyLen}%
                @{}p{\linewidth-\the\MyLen}@{}}
%\begin{tabular}{@{}ll@{}}
\texttt{fullpage}  &  Use 1 inch margins. \\
\texttt{anysize}   &  Set margins: \verb!\marginsize{!\textit{l}%
                        \verb!}{!\textit{r}\verb!}{!\textit{t}%
                        \verb!}{!\textit{b}\verb!}!.            \\
\texttt{multicol}  &  Use $n$ columns: 
                        \verb!\begin{multicols}{!$n$\verb!}!.   \\
\texttt{latexsym}  &  Use \LaTeX\ symbol font. \\
\texttt{graphicx}  &  Show image:
                        \verb!\includegraphics[width=!%
                        \textit{x}\verb!]{!%
                        \textit{file}\verb!}!. \\
\texttt{url}       & Insert URL: \verb!\url{!%
                        \textit{http://\ldots}%
                        \verb!}!.
\end{tabular}

Use before \verb!\begin{document}!. 
Usage: \verb!\usepackage{!\textit{package}\verb!}!


\subsection{Title}
\settowidth{\MyLen}{\texttt{.author.text.} }
\begin{tabular}{@{}p{\the\MyLen}%
                @{}p{\linewidth-\the\MyLen}@{}}
\verb!\author{!\textit{text}\verb!}! & Author of document. \\
\verb!\title{!\textit{text}\verb!}!  & Title of document. \\
\verb!\date{!\textit{text}\verb!}!   & Date. \\
\end{tabular}

These commands go before \verb!\begin{document}!.  The declaration
\verb!\maketitle! goes at the top of the document.

\subsection{Miscellaneous}
\settowidth{\MyLen}{\texttt{.pagestyle.empty.} }
\begin{tabular}{@{}p{\the\MyLen}%
                @{}p{\linewidth-\the\MyLen}@{}}
\verb!\pagestyle{empty}!     &   Empty header, footer
                                 and no page numbers. \\
\verb!\tableofcontents!      &   Add a table of contents here. \\

\end{tabular}



\section{Document structure}
\begin{multicols}{2}
\verb!\part{!\textit{title}\verb!}!  \\
\verb!\chapter{!\textit{title}\verb!}!  \\
\verb!\section{!\textit{title}\verb!}!  \\
\verb!\subsection{!\textit{title}\verb!}!  \\
\verb!\subsubsection{!\textit{title}\verb!}!  \\
\verb!\paragraph{!\textit{title}\verb!}!  \\
\verb!\subparagraph{!\textit{title}\verb!}!
\end{multicols}
{\raggedright
Use \verb!\setcounter{secnumdepth}{!$x$\verb!}! suppresses heading
numbers of depth $>x$, where \verb!chapter! has depth 0.
Use a \texttt{*}, as in \verb!\section*{!\textit{title}\verb!}!,
to not number a particular item---these items will also not appear
in the table of contents.
}

\subsection{Text environments}
\settowidth{\MyLen}{\texttt{.begin.quotation.}}
\begin{tabular}{@{}p{\the\MyLen}%
                @{}p{\linewidth-\the\MyLen}@{}}
\verb!\begin{comment}!    &  Comment (not printed). Requires \texttt{verbatim} package. \\
\verb!\begin{quote}!      &  Indented quotation block. \\
\verb!\begin{quotation}!  &  Like \texttt{quote} with indented paragraphs. \\
\verb!\begin{verse}!      &  Quotation block for verse.
\end{tabular}

\subsection{Lists}
\settowidth{\MyLen}{\texttt{.begin.description.}}
\begin{tabular}{@{}p{\the\MyLen}%
                @{}p{\linewidth-\the\MyLen}@{}}
\verb!\begin{enumerate}!        &  Numbered list. \\
\verb!\begin{itemize}!          &  Bulleted list. \\
\verb!\begin{description}!      &  Description list. \\
\verb!\item! \textit{text}      &  Add an item. \\
\verb!\item[!\textit{x}\verb!]! \textit{text}
                                &  Use \textit{x} instead of normal
                        bullet or number.  Required for descriptions. \\
\end{tabular}




\subsection{References}
\settowidth{\MyLen}{\texttt{.pageref.marker..}}
\begin{tabular}{@{}p{\the\MyLen}%
                @{}p{\linewidth-\the\MyLen}@{}}
\verb!\label{!\textit{marker}\verb!}!   &  Set a marker for cross-reference, 
                          often of the form \verb!\label{sec:item}!. \\
\verb!\ref{!\textit{marker}\verb!}!   &  Give section/body number of marker. \\
\verb!\pageref{!\textit{marker}\verb!}! &  Give page number of marker. \\
\verb!\footnote{!\textit{text}\verb!}!  &  Print footnote at bottom of page. \\
\end{tabular}


\subsection{Floating bodies}
\settowidth{\MyLen}{\texttt{.begin.equation..place.}}
\begin{tabular}{@{}p{\the\MyLen}%
                @{}p{\linewidth-\the\MyLen}@{}}
\verb!\begin{table}[!\textit{place}\verb!]!     &  Add numbered table. \\
\verb!\begin{figure}[!\textit{place}\verb!]!    &  Add numbered figure. \\
\verb!\begin{equation}[!\textit{place}\verb!]!  &  Add numbered equation. \\
\verb!\caption{!\textit{text}\verb!}!           &  Caption for the body. \\
\end{tabular}

The \textit{place} is a list valid placements for the body.  \texttt{t}=top,
\texttt{h}=here, \texttt{b}=bottom, \texttt{p}=separate page, \texttt{!}=place even if ugly.  Captions and label markers should be within the environment.

%---------------------------------------------------------------------------

\section{Text properties}

\subsection{Font face}
\newcommand{\FontCmd}[3]{\PBS\verb!\#1{!\textit{text}\verb!}!  \> %
                         \verb!{\#2 !\textit{text}\verb!}! \> %
                         \#1{#3}}
\begin{tabular}{@{}l@{}l@{}l@{}}
\textit{Command} & \textit{Declaration} & \textit{Effect} \\
\verb!\textrm{!\textit{text}\verb!}!                    & %
        \verb!{\rmfamily !\textit{text}\verb!}!               & %
        \textrm{Roman family} \\
\verb!\textsf{!\textit{text}\verb!}!                    & %
        \verb!{\sffamily !\textit{text}\verb!}!               & %
        \textsf{Sans serif family} \\
\verb!\texttt{!\textit{text}\verb!}!                    & %
        \verb!{\ttfamily !\textit{text}\verb!}!               & %
        \texttt{Typewriter family} \\
\verb!\textmd{!\textit{text}\verb!}!                    & %
        \verb!{\mdseries !\textit{text}\verb!}!               & %
        \textmd{Medium series} \\
\verb!\textbf{!\textit{text}\verb!}!                    & %
        \verb!{\bfseries !\textit{text}\verb!}!               & %
        \textbf{Bold series} \\
\verb!\textup{!\textit{text}\verb!}!                    & %
        \verb!{\upshape !\textit{text}\verb!}!               & %
        \textup{Upright shape} \\
\verb!\textit{!\textit{text}\verb!}!                    & %
        \verb!{\itshape !\textit{text}\verb!}!               & %
        \textit{Italic shape} \\
\verb!\textsl{!\textit{text}\verb!}!                    & %
        \verb!{\slshape !\textit{text}\verb!}!               & %
        \textsl{Slanted shape} \\
\verb!\textsc{!\textit{text}\verb!}!                    & %
        \verb!{\scshape !\textit{text}\verb!}!               & %
        \textsc{Small Caps shape} \\
\verb!\emph{!\textit{text}\verb!}!                      & %
        \verb!{\em !\textit{text}\verb!}!               & %
        \emph{Emphasized} \\
\verb!\textnormal{!\textit{text}\verb!}!                & %
        \verb!{\normalfont !\textit{text}\verb!}!       & %
        \textnormal{Document font} \\
\verb!\underline{!\textit{text}\verb!}!                 & %
                                                        & %
        \underline{Underline}
\end{tabular}

The command (t\textit{tt}t) form handles spacing better than the
declaration (t{\itshape tt}t) form.

\subsection{Font size}
\setlength{\columnsep}{14pt} % Need to move columns apart a little
\begin{multicols}{2}
\begin{tabbing}
\verb!\footnotesize!          \= \kill
\verb!\tiny!                  \>  \tiny{tiny} \\
\verb!\scriptsize!            \>  \scriptsize{scriptsize} \\
\verb!\footnotesize!          \>  \footnotesize{footnotesize} \\
\verb!\small!                 \>  \small{small} \\
\verb!\normalsize!            \>  \normalsize{normalsize} \\
\verb!\large!                 \>  \large{large} \\
\verb!\Large!                 \=  \Large{Large} \\  % Tab hack for new column
\verb!\LARGE!                 \>  \LARGE{LARGE} \\
\verb!\huge!                  \>  \huge{huge} \\
\verb!\Huge!                  \>  \Huge{Huge}
\end{tabbing}
\end{multicols}
\setlength{\columnsep}{1pt} % Set column separation back

These are declarations and should be used in the form
\verb!{\small! \ldots\verb!}!, or without braces to affect the entire
document.


\subsection{Verbatim text}

\settowidth{\MyLen}{\texttt{.begin.verbatim..} }
\begin{tabular}{@{}p{\the\MyLen}%
                @{}p{\linewidth-\the\MyLen}@{}}
\verb@\begin{verbatim}@ & Verbatim environment. \\
\verb@\begin{verbatim*}@ & Spaces are shown as \verb*@ @. \\
\verb@\verb!text!@ & Text between the delimiting characters (in this case %
                      `\texttt{!}') is verbatim.
\end{tabular}


\subsection{Justification}
\begin{tabular}{@{}ll@{}}
\textit{Environment}  &  \textit{Declaration}  \\
\verb!\begin{center}!      & \verb!\centering!  \\
\verb!\begin{flushleft}!  & \verb!\raggedright! \\
\verb!\begin{flushright}! & \verb!\raggedleft!  \\
\end{tabular}

\subsection{Miscellaneous}
\verb!\linespread{!$x$\verb!}! changes the line spacing by the
multiplier $x$.





\section{Text-mode symbols}

\subsection{Symbols}
\begin{tabular}{@{}l@{\hspace{1em}}l@{\hspace{2em}}l@{\hspace{1em}}l@{\hspace{2em}}l@{\hspace{1em}}l@{\hspace{2em}}l@{\hspace{1em}}l@{}}
\&              &  \verb!\&! &
\_              &  \verb!\_! &
\ldots          &  \verb!\ldots! &
\textbullet     &  \verb!\textbullet! \\
\$              &  \verb!\$! &
\^{}            &  \verb!\^{}! &
\textbar        &  \verb!\textbar! &
\textbackslash  &  \verb!\textbackslash! \\
\%              &  \verb!\%! &
\~{}            &  \verb!\~{}! &
\#              &  \verb!\#! &
\S              &  \verb!\S! \\
\end{tabular}

\subsection{Accents}
\begin{tabular}{@{}l@{\ }l|l@{\ }l|l@{\ }l|l@{\ }l|l@{\ }l@{}}
\`o   & \verb!\`o! &
\'o   & \verb!\'o! &
\^o   & \verb!\^o! &
\~o   & \verb!\~o! &
\=o   & \verb!\=o! \\
\.o   & \verb!\.o! &
\"o   & \verb!\"o! &
\c o  & \verb!\c o! &
\v o  & \verb!\v o! &
\H o  & \verb!\H o! \\
\c c  & \verb!\c c! &
\d o  & \verb!\d o! &
\b o  & \verb!\b o! &
\t oo & \verb!\t oo! &
\oe   & \verb!\oe! \\
\OE   & \verb!\OE! &
\ae   & \verb!\ae! &
\AE   & \verb!\AE! &
\aa   & \verb!\aa! &
\AA   & \verb!\AA! \\
\o    & \verb!\o! &
\O    & \verb!\O! &
\l    & \verb!\l! &
\L    & \verb!\L! &
\i    & \verb!\i! \\
\j    & \verb!\j! &
!`    & \verb!~`! &
?`    & \verb!?`! &
\end{tabular}


\subsection{Delimiters}
\begin{tabular}{@{}l@{\ }ll@{\ }ll@{\ }ll@{\ }ll@{\ }ll@{\ }l@{}}
`       & \verb!`!  &
``      & \verb!``! &
\{      & \verb!\{! &
\lbrack & \verb![! &
(       & \verb!(! &
\textless  &  \verb!\textless! \\
'       & \verb!'!  &
''      & \verb!''! &
\}      & \verb!\}! &
\rbrack & \verb!]! &
)       & \verb!)! &
\textgreater  &  \verb!\textgreater! \\
\end{tabular}

\subsection{Dashes}
\begin{tabular}{@{}llll@{}}
\textit{Name} & \textit{Source} & \textit{Example} & \textit{Usage} \\
hyphen  & \verb!-!   & X-ray          & In words. \\
en-dash & \verb!--!  & 1--5           & Between numbers. \\
em-dash & \verb!---! & Yes---or no?    & Punctuation.
\end{tabular}


\subsection{Line and page breaks}
\settowidth{\MyLen}{\texttt{.pagebreak} }
\begin{tabular}{@{}p{\the\MyLen}%
                @{}p{\linewidth-\the\MyLen}@{}}
\verb!\\!          &  Begin new line without new paragraph.  \\
\verb!\\*!         &  Prohibit pagebreak after linebreak. \\
\verb!\kill!       &  Don't print current line. \\
\verb!\pagebreak!  &  Start new page. \\
\verb!\noindent!   &  Do not indent current line.
\end{tabular}


\subsection{Miscellaneous}
\settowidth{\MyLen}{\texttt{.rule.w..h.} }
\begin{tabular}{@{}p{\the\MyLen}%
                @{}p{\linewidth-\the\MyLen}@{}}
\verb!\today!  &  \today. \\
\verb!$\sim$!  &  Prints $\sim$ instead of \verb!\~{}!, which makes \~{}. \\
\verb!~!       &  Space, disallow linebreak (\verb!W.J.~Clinton!). \\
\verb!\@.!     &  Indicate that the \verb!.! ends a sentence when following
                        an uppercase letter. \\
\verb!\hspace{!$l$\verb!}! & Horizontal space of length $l$
                                (Ex: $l=\mathtt{20pt}$). \\
\verb!\vspace{!$l$\verb!}! & Vertical space of length $l$. \\
\verb!\rule{!$w$\verb!}{!$h$\verb!}! & Line of width $w$ and height $h$. \\
\end{tabular}



\section{Tabular environments}

\subsection{\texttt{tabbing} environment}
\begin{tabular}{@{}l@{\hspace{1.5ex}}l@{\hspace{10ex}}l@{\hspace{1.5ex}}l@{}}
\verb!\=!  &   Set tab stop. &
\verb!\>!  &   Go to tab stop.
\end{tabular}

Tab stops can be set on ``invisible'' lines with \verb!\kill!
at the end of the line.  Normally \verb!\\! is used to separate lines.


\subsection{\texttt{tabular} environment}
\verb!\begin{array}[!\textit{pos}\verb!]{!\textit{cols}\verb!}!   \\
\verb!\begin{tabular}[!\textit{pos}\verb!]{!\textit{cols}\verb!}! \\
\verb!\begin{tabular*}{!\textit{width}\verb!}[!\textit{pos}\verb!]{!\textit{cols}\verb!}!


\subsubsection{\texttt{tabular} column specification}
\settowidth{\MyLen}{\texttt{p}\{\textit{width}\} \ }
\begin{tabular}{@{}p{\the\MyLen}@{}p{\linewidth-\the\MyLen}@{}}
\texttt{l}    &   Left-justified column.  \\
\texttt{c}    &   Centered column.  \\
\texttt{r}    &   Right-justified column. \\
\verb!p{!\textit{width}\verb!}!  &  Same as %
                              \verb!\parbox[t]{!\textit{width}\verb!}!. \\ 
\verb!@{!\textit{decl}\verb!}!   &  Insert \textit{decl} instead of
                                    inter-column space. \\
\verb!|!      &   Inserts a vertical line between columns. 
\end{tabular}


\subsubsection{\texttt{tabular} elements}
\settowidth{\MyLen}{\texttt{.cline.x-y..}}
\begin{tabular}{@{}p{\the\MyLen}@{}p{\linewidth-\the\MyLen}@{}}
\verb!\hline!           &  Horizontal line between rows.  \\
\verb!\cline{!$x$\verb!-!$y$\verb!}!  &
                        Horizontal line across columns $x$ through $y$. \\
\verb!\multicolumn{!\textit{n}\verb!}{!\textit{cols}\verb!}{!\textit{text}\verb!}! \\
        &  A cell that spans \textit{n} columns, with \textit{cols} column specification.
\end{tabular}

\section{Math mode}
For inline math, use \verb!\(...\)! or \verb!$...$!.
For displayed math, use \verb!\[...\]! or \verb!\begin{equation}!.

\begin{tabular}{@{}l@{\hspace{1em}}l@{\hspace{2em}}l@{\hspace{1em}}l@{}}
Superscript$^{x}$       &
\verb!^{x}!             &  
Subscript$_{x}$         &
\verb!_{x}!             \\  
$\frac{x}{y}$           &
\verb!\frac{x}{y}!      &  
$\sum_{k=1}^n$          &
\verb!\sum_{k=1}^n!     \\  
$\sqrt[n]{x}$           &
\verb!\sqrt[n]{x}!      &  
$\prod_{k=1}^n$         &
\verb!\prod_{k=1}^n!    \\ 
\end{tabular}

\subsection{Math-mode symbols}

% The ordering of these symbols is slightly odd.  This is because I had to put all the
% long pieces of text in the same column (the right) for it all to fit properly.
% Otherwise, it wouldn't be possible to fit four columns of symbols here.

\begin{tabular}{@{}l@{\hspace{1ex}}l@{\hspace{1em}}l@{\hspace{1ex}}l@{\hspace{1em}}l@{\hspace{1ex}} l@{\hspace{1em}}l@{\hspace{1ex}}l@{}}
$\leq$          &  \verb!\leq!  &
$\geq$          &  \verb!\geq!  &
$\neq$          &  \verb!\neq!  &
$\approx$       &  \verb!\approx!  \\
$\times$        &  \verb!\times!  &
$\div$          &  \verb!\div!  &
$\pm$           & \verb!\pm!  &
$\cdot$         &  \verb!\cdot!  \\
$^{\circ}$      & \verb!^{\circ}! &
$\circ$         &  \verb!\circ!  &
$\prime$        & \verb!\prime!  &
$\cdots$        &  \verb!\cdots!  \\
$\infty$        & \verb!\infty!  &
$\neg$          & \verb!\neg!  &
$\wedge$        & \verb!\wedge!  &
$\vee$          & \verb!\vee!  \\
$\supset$       & \verb!\supset!  &
$\forall$       & \verb!\forall!  &
$\in$           & \verb!\in!  &
$\rightarrow$   &  \verb!\rightarrow! \\
$\subset$       & \verb!\subset!  &
$\exists$       & \verb!\exists!  &
$\notin$        & \verb!\notin!  &
$\Rightarrow$   &  \verb!\Rightarrow! \\
$\cup$          & \verb!\cup!  &
$\cap$          & \verb!\cap!  &
$\mid$          & \verb!\mid!  &
$\Leftrightarrow$   &  \verb!\Leftrightarrow! \\
$\dot a$        & \verb!\dot a!  &
$\hat a$        & \verb!\hat a!  &
$\bar a$        & \verb!\bar a!  &
$\tilde a$      & \verb!\tilde a!  \\

$\alpha$        &  \verb!\alpha!  &
$\beta$         &  \verb!\beta!  &
$\gamma$        &  \verb!\gamma!  &
$\delta$        &  \verb!\delta!  \\
$\epsilon$      &  \verb!\epsilon!  &
$\zeta$         &  \verb!\zeta!  &
$\eta$          &  \verb!\eta!  &
$\varepsilon$   &  \verb!\varepsilon!  \\
$\theta$        &  \verb!\theta!  &
$\iota$         &  \verb!\iota!  &
$\kappa$        &  \verb!\kappa!  &
$\vartheta$     &  \verb!\vartheta!  \\
$\lambda$       &  \verb!\lambda!  &
$\mu$           &  \verb!\mu!  &
$\nu$           &  \verb!\nu!  &
$\xi$           &  \verb!\xi!  \\
$\pi$           &  \verb!\pi!  &
$\rho$          &  \verb!\rho!  &
$\sigma$        &  \verb!\sigma!  &
$\tau$          &  \verb!\tau!  \\
$\upsilon$      &  \verb!\upsilon!  &
$\phi$          &  \verb!\phi!  &
$\chi$          &  \verb!\chi!  &
$\psi$          &  \verb!\psi!  \\
$\omega$        &  \verb!\omega!  &
$\Gamma$        &  \verb!\Gamma!  &
$\Delta$        &  \verb!\Delta!  &
$\Theta$        &  \verb!\Theta!  \\
$\Lambda$       &  \verb!\Lambda!  &
$\Xi$           &  \verb!\Xi!  &
$\Pi$           &  \verb!\Pi!  &
$\Sigma$        &  \verb!\Sigma!  \\
$\Upsilon$      &  \verb!\Upsilon!  &
$\Phi$          &  \verb!\Phi!  &
$\Psi$          &  \verb!\Psi!  &
$\Omega$        &  \verb!\Omega!  
\end{tabular}
\footnotesize

%\subsection{Special symbols}
%\begin{tabular}{@{}ll@{}}
%$^{\circ}$  &  \verb!^{\circ}! Ex: $22^{\circ}\mathrm{C}$: \verb!$22^{\circ}\mathrm{C}$!.
%\end{tabular}

\section{Bibliography and citations}
When using \BibTeX, you need to run \texttt{latex}, \texttt{bibtex},
and \texttt{latex} twice more to resolve dependencies.

\subsection{Citation types}
\settowidth{\MyLen}{\texttt{.shortciteN.key..}}
\begin{tabular}{@{}p{\the\MyLen}@{}p{\linewidth-\the\MyLen}@{}}
\verb!\cite{!\textit{key}\verb!}!       &
        Full author list and year. (Watson and Crick 1953) \\
\verb!\citeA{!\textit{key}\verb!}!      &
        Full author list. (Watson and Crick) \\
\verb!\citeN{!\textit{key}\verb!}!      &
        Full author list and year. Watson and Crick (1953) \\
\verb!\shortcite{!\textit{key}\verb!}!  &
        Abbreviated author list and year. ? \\
\verb!\shortciteA{!\textit{key}\verb!}! &
        Abbreviated author list. ? \\
\verb!\shortciteN{!\textit{key}\verb!}! &
        Abbreviated author list and year. ? \\
\verb!\citeyear{!\textit{key}\verb!}!   &
        Cite year only. (1953) \\
\end{tabular}

All the above have an \texttt{NP} variant without parentheses;
Ex. \verb!\citeNP!.


\subsection{\BibTeX\ entry types}
\settowidth{\MyLen}{\texttt{.mastersthesis.}}
\begin{tabular}{@{}p{\the\MyLen}@{}p{\linewidth-\the\MyLen}@{}}
\verb!@article!         &  Journal or magazine article. \\
\verb!@book!            &  Book with publisher. \\
\verb!@booklet!         &  Book without publisher. \\
\verb!@conference!      &  Article in conference proceedings. \\
\verb!@inbook!          &  A part of a book and/or range of pages. \\
\verb!@incollection!    &  A part of book with its own title. \\
%\verb!@manual!          &  Technical documentation. \\
%\verb!@mastersthesis!   &  Master's thesis. \\
\verb!@misc!            &  If nothing else fits. \\
\verb!@phdthesis!       &  PhD. thesis. \\
\verb!@proceedings!     &  Proceedings of a conference. \\
\verb!@techreport!      &  Tech report, usually numbered in series. \\
\verb!@unpublished!     &  Unpublished. \\
\end{tabular}

\subsection{\BibTeX\ fields}
\settowidth{\MyLen}{\texttt{organization.}}
\begin{tabular}{@{}p{\the\MyLen}@{}p{\linewidth-\the\MyLen}@{}}
\verb!address!         &  Address of publisher.  Not necessary for major
                                publishers.  \\
\verb!author!           &  Names of authors, of format .... \\
\verb!booktitle!        &  Title of book when part of it is cited. \\
\verb!chapter!          &  Chapter or section number. \\
\verb!edition!          &  Edition of a book. \\
\verb!editor!           &  Names of editors. \\
\verb!institution!      &  Sponsoring institution of tech.\ report. \\
\verb!journal!          &  Journal name. \\
\verb!key!              &  Used for cross ref.\ when no author. \\
\verb!month!            &  Month published. Use 3-letter abbreviation. \\
\verb!note!             &  Any additional information. \\
\verb!number!           &  Number of journal or magazine. \\
\verb!organization!     &  Organization that sponsors a conference. \\
\verb!pages!            &  Page range (\verb!2,6,9--12!). \\
\verb!publisher!        &  Publisher's name. \\
\verb!school!           &  Name of school (for thesis). \\
\verb!series!           &  Name of series of books. \\
\verb!title!            &  Title of work. \\
\verb!type!             &  Type of tech.\ report, ex. ``Research Note''. \\
\verb!volume!           &  Volume of a journal or book. \\
\verb!year!             &  Year of publication. \\
\end{tabular}
Not all fields need to be filled.  See example below.

\subsection{Common \BibTeX\ style files}
\begin{tabular}{@{}l@{\hspace{1em}}l@{\hspace{3em}}l@{\hspace{1em}}l@{}}
\verb!abbrv!    &  Standard &
\verb!abstract! &  \texttt{alpha} with abstract \\
\verb!alpha!    &  Standard &
\verb!apa!      &  APA \\
\verb!plain!    &  Standard &
\verb!unsrt!    &  Unsorted \\
\end{tabular}

The \LaTeX\ document should have the following two lines just before
\verb!\end{document}!, where \verb!bibfile.bib! is the name of the
\BibTeX\ file.
\begin{verbatim}
\bibliographystyle{plain}
\bibliography{bibfile}
\end{verbatim}

\subsection{\BibTeX\ example}
The \BibTeX\ database goes in a file called
\textit{file}\texttt{.bib}, which is processed with \verb!bibtex file!. 
\begin{verbatim}
@String{N = {Na\-ture}}
@Article{WC:1953,
  author  = {James Watson and Francis Crick},
  title   = {A structure for Deoxyribose Nucleic Acid},
  journal = N,
  volume  = {171},
  pages   = {737},
  year    = 1953
}
\end{verbatim}


\section{Sample \LaTeX\ document}
\begin{verbatim}
\documentclass[11pt]{article}
\usepackage{fullpage}
\title{Template}
\author{Name}
\begin{document}
\maketitle

\section{section}
\subsection*{subsection without number}
text \textbf{bold text} text. Some math: $2+2=5$
\subsection{subsection}
text \emph{emphasized text} text. \cite{WC:1953}
discovered the structure of DNA.

A table:
\begin{table}[!th]
\begin{tabular}{|l|c|r|}
\hline
first  &  row  &  data \\
second &  row  &  data \\
\hline
\end{tabular}
\caption{This is the caption}
\label{ex:table}
\end{table}

The table is numbered \ref{ex:table}.
\end{document}
\end{verbatim}



\rule{0.3\linewidth}{0.25pt}
\scriptsize

Copyright \copyright\ 2012 Winston Chang

http://www.stdout.org/$\sim$winston/latex/


\end{multicols}
\end{document}
