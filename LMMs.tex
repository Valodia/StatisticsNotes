\documentclass[10pt,landscape]{article}

\usepackage{multicol}
\usepackage{calc}
\usepackage{ifthen}
\usepackage[landscape]{geometry}

\usepackage{amsmath}
\usepackage{amssymb}
\usepackage{gauss}

%% taken from http://brunoj.wordpress.com/2009/10/08/latex-the-framed-minipage/
\newsavebox{\fmbox}
\newenvironment{fmpage}[1]
{\begin{lrbox}{\fmbox}\begin{minipage}{#1}}
{\end{minipage}\end{lrbox}\fbox{\usebox{\fmbox}}}

\usepackage{mathtools}
\makeatletter
 
\newcommand{\explain}[2]{\underset{\mathclap{\overset{\uparrow}{#2}}}{#1}}
\newcommand{\explainup}[2]{\overset{\mathclap{\underset{\downarrow}{#2}}}{#1}}
 
\makeatother

% To make this come out properly in landscape mode, do one of the following
% 1.
%  pdflatex latexsheet.tex
%
% 2.
%  latex latexsheet.tex
%  dvips -P pdf  -t landscape latexsheet.dvi
%  ps2pdf latexsheet.ps


% If you're reading this, be prepared for confusion.  Making this was
% a learning experience for me, and it shows.  Much of the placement
% was hacked in; if you make it better, let me know...


% 2008-04
% Changed page margin code to use the geometry package. Also added code for
% conditional page margins, depending on paper size. Thanks to Uwe Ziegenhagen
% for the suggestions.

% 2006-08
% Made changes based on suggestions from Gene Cooperman. <gene at ccs.neu.edu>


% To Do:
% \listoffigures \listoftables
% \setcounter{secnumdepth}{0}


% This sets page margins to .5 inch if using letter paper, and to 1cm
% if using A4 paper. (This probably isn't strictly necessary.)
% If using another size paper, use default 1cm margins.
\ifthenelse{\lengthtest { \paperwidth = 11in}}
	{ \geometry{top=.5in,left=.5in,right=.5in,bottom=.5in} }
	{\ifthenelse{ \lengthtest{ \paperwidth = 297mm}}
		{\geometry{top=1cm,left=1cm,right=1cm,bottom=1cm} }
		{\geometry{top=1cm,left=1cm,right=1cm,bottom=1cm} }
	}

% Turn off header and footer
\pagestyle{empty}
 

% Redefine section commands to use less space
\makeatletter
\renewcommand{\section}{\@startsection{section}{1}{0mm}%
                                {-1ex plus -.5ex minus -.2ex}%
                                {0.5ex plus .2ex}%x
                                {\normalfont\large\bfseries}}
\renewcommand{\subsection}{\@startsection{subsection}{2}{0mm}%
                                {-1explus -.5ex minus -.2ex}%
                                {0.5ex plus .2ex}%
                                {\normalfont\normalsize\bfseries}}
\renewcommand{\subsubsection}{\@startsection{subsubsection}{3}{0mm}%
                                {-1ex plus -.5ex minus -.2ex}%
                                {1ex plus .2ex}%
                                {\normalfont\small\bfseries}}
\makeatother

% Define BibTeX command
\def\BibTeX{{\rm B\kern-.05em{\sc i\kern-.025em b}\kern-.08em
    T\kern-.1667em\lower.7ex\hbox{E}\kern-.125emX}}

% Don't print section numbers
%\setcounter{secnumdepth}{0}



\setlength{\parindent}{0pt}
\setlength{\parskip}{0pt plus 0.5ex}





% -----------------------------------------------------------------------

\usepackage{Sweave}
\begin{document}

\Sconcordance{concordance:LMMlong.tex:LMMlong.Rnw:%
1 32 1 1 0 3 1 1 12}
\Sconcordance{concordance:LMMlong.tex:./LMMnotes.Rnw:ofs 38:%
1 8 1 1 2 1 0 2 1 17 0 1 2 40 %
1 1 2 18 0 1 2 17 1 1 2 1 0 1 %
1 16 0 1 2 19 0 1 2 17 1 1 2 9 %
0 1 2 47 1 1 2 7 0 1 2 4 1 1 2 %
1 0 1 1 17 0 1 2 20 1 1 2 9 0 %
1 2 59 1 1 3 5 0 1 2 74 1 1 3 %
2 0 1 2 13 0 1 6 1 2 13 0 1 2 %
3 1 1 2 10 0 1 2 12 1 1 4 6 0 %
2 2 14 0 1 2 12 1 1 2 14 0 1 2 %
36 1 1 2 7 0 1 1 17 0 1 4 31 1 %
1 5 15 0 1 1 13 0 1 2 58 1 1 3 %
32 0 1 3 16 0 1 3 17 0 1 5 3 0 %
1 5 8 0 1 4 10 0 1 3 10 0 1 3 %
10 0 1 3 7 0 1 3 6 0 1 3 15 0 %
1 1 16 0 1 2 10 1 1 2 4 0 1 2 %
12 1 1 2 1 0 1 1 2 2 2 1 6 0 1 %
3 7 0 1 2 2 1 1 2 1 0 2 1 8 0 %
1 2 4 1 1 2 1 0 1 1 1 2 1 1 2 %
2 1 1 1 2 3 0 1 2 2 1 1 2 1 0 %
1 2 3 1 1 2 1 3 1 0 1 2 1 0 1 %
3 7 0 1 2 15 1 1 2 1 0 1 2 1 0 %
1 2 1 0 2 1 2 2 1 1 3 2 3 1 5 %
0 1 3 1 0 1 2 1 0 1 2 1 0 2 1 %
2 2 1 1 2 2 3 0 1 2 24 1}
\Sconcordance{concordance:LMMlong.tex:LMMlong.Rnw:ofs 1106:%
50 2 1}


\setkeys{Gin}{width=\textwidth}


\raggedright
\footnotesize
\begin{multicols}{3}


% multicol parameters
% These lengths are set only within the two main columns
%\setlength{\columnseprule}{0.25pt}
\setlength{\premulticols}{1pt}
\setlength{\postmulticols}{1pt}
\setlength{\multicolsep}{1pt}
\setlength{\columnsep}{2pt}

\begin{center}
     \normalsize{Linear Mixed Models Summary Sheet} \\
    \footnotesize{
    Compiled by: Shravan Vasishth (vasishth@uni-potsdam.de)\\
    Version dated: \today}
\end{center}


%\textbf{Background for this summary sheet}

%This is a summary for myself, for reviewing the material I learnt while doing an MSc-level course in statistics; your mileage may vary.
%%input the notes:


These notes summarize the lecture notes from the Linear Modelling course at Sheffield's School of Mathematics and Statistics, MSc degree programme. The original notes were written by Dr.\ Jeremy Oakley. This summary is completely derived from these notes and from other MSc sources. Any errors are most probably mine.

Everything is in matrix form unless a lower case letter with a subscript (such as $x_i$) is used (even there, I might deviate from this convention if I need to index sub-matrices; it's best to look at the context to decide what is meant).

\section{Some basic types of linear mixed model and their variance components}

\subsection{Varying intercepts model}

\begin{Schunk}
\begin{Sinput}
> library(lme4)
> fm1<-lmer(wear~material+(1|Subject),BHHshoes)
> ranef(fm1)
\end{Sinput}
\begin{Soutput}
$Subject
   (Intercept)
1      2.74820
2     -2.32081
3      0.21369
4      3.39425
5      0.41248
6     -4.30866
7     -1.17780
8      0.21369
9     -1.77415
10     2.59911
\end{Soutput}
\end{Schunk}

The model is:

\begin{equation}
Y_{ijk} = \beta_j + b_{i}+\epsilon_{ijk}
\end{equation}

\noindent
$i=1,\dots,10$ is subject id, $j=1,2$ is the factor level, $k$ is the number of replicates (here 1).
$b_i \sim N(0,\sigma_b^2), \epsilon_{ijk}\sim N(0,\sigma^2)$.

The general form for any model in this case is:

\begin{equation}
\begin{pmatrix}
Y_{i1}\\
Y_{i2}
\end{pmatrix}
\sim
N\left(
\begin{pmatrix}
\beta_1\\
\beta_2\\
\end{pmatrix}
,
V
\right)
\end{equation}

where $V =\begin{pmatrix}
\sigma_b^2 + \sigma^2 & \sigma_b^2\\
\sigma_b^2 & \sigma_b^2 + \sigma^2\\
\end{pmatrix}
=
\begin{pmatrix}
\sigma^2_{b} + \sigma^2  &  \rho\sigma_{b}\sigma_{b}\\
\rho\sigma_{b}\sigma_{b} & \sigma^2_{b}+\sigma^2  \\       
\end{pmatrix}$.

We can recover these variance components as follows:

\begin{Schunk}
\begin{Sinput}
> VarCorr(fm1)
\end{Sinput}
\begin{Soutput}
$Subject
            (Intercept)
(Intercept)      6.1009
attr(,"stddev")
(Intercept) 
       2.47 
attr(,"correlation")
            (Intercept)
(Intercept)           1

attr(,"sc")
[1] 0.27376
\end{Soutput}
\end{Schunk}

$\hat{V}$ is therefore:

\begin{equation}
\begin{pmatrix}
\hat{\sigma}^2_{b} + \hat{\sigma}^2  &  \hat{\rho}\hat{\sigma}_{b}\hat{\sigma}_{b}\\
\hat{\rho}\hat{\sigma}_{b}\hat{\sigma}_{b} & \hat{\sigma}^2_{b}+\hat{\sigma}^2  \\       
\end{pmatrix}=
\begin{pmatrix}
2.47^2 + 0.27376^2 & 2.47^2\\
2.47^2 & 2.47^2 + 0.27376^2\\
\end{pmatrix}
\end{equation}

Note: $\hat{\rho}=1$ because the off-diagonal is $2.47^2=1\times 2.47 \times 2.47$. But this correlation is not estimated in the varying intercepts model.

\subsection{Varying intercepts and slopes (with correlation)}

\begin{Schunk}
\begin{Sinput}
> fm2<-lmer(wear~material+(1+material|Subject),BHHshoes)
> ranef(fm2)
\end{Sinput}
\begin{Soutput}
$Subject
   (Intercept)  materialB
1      2.71318  0.0752088
2     -2.28776 -0.0634150
3      0.21003  0.0058217
4      3.34435  0.0927024
5      0.41145  0.0114069
6     -4.25374 -0.1179130
7     -1.16241 -0.0322216
8      0.21137  0.0058593
9     -1.74925 -0.0484882
10     2.56278  0.0710387
\end{Soutput}
\begin{Sinput}
> VarCorr(fm2)
\end{Sinput}
\begin{Soutput}
$Subject
            (Intercept) materialB
(Intercept)     5.93634 0.1645525
materialB       0.16455 0.0045617
attr(,"stddev")
(Intercept)   materialB 
    2.43646     0.06754 
attr(,"correlation")
            (Intercept) materialB
(Intercept)     1.00000   0.99996
materialB       0.99996   1.00000

attr(,"sc")
[1] 0.26956
\end{Soutput}
\end{Schunk}

The model is 
\begin{equation}
Y_{ijk} = \beta_j + b_{ij}+\epsilon_{ijk}
\end{equation}

\noindent
$b_{ij}\sim N(0,\sigma_b)$. The variance $\sigma_b$ must be a $2\times 2$ matrix:

\begin{equation}
\begin{pmatrix}
\sigma_1^2 & \rho \sigma_1 \sigma_2\\
\rho \sigma_1 \sigma_2 & \sigma_2^2\\
\end{pmatrix}
\end{equation}

We can recover this from the random effects:

\begin{Schunk}
\begin{Sinput}
> var(ranef(fm2)$Subject)
\end{Sinput}
\begin{Soutput}
            (Intercept) materialB
(Intercept)     5.90124 0.1635798
materialB       0.16358 0.0045344
\end{Soutput}
\end{Schunk}

Note that $1\times \sqrt{5.90124}\times \sqrt{0.0045344}=0.16358$, which is how we get that $\hat{\rho}=1$.

The general form for the model is:

\begin{equation}
\begin{pmatrix}
Y_{i1}\\
Y_{i2}
\end{pmatrix}
\sim
N\left( 
\begin{pmatrix}
\beta_1\\
\beta_2\\
\end{pmatrix}
,
V
\right)
\end{equation}

where 

\begin{equation}
V =
%\begin{pmatrix}
%\sigma_b^2 + \sigma^2 & \sigma_b^2\\
%\sigma_b^2 & \sigma_b^2 + \sigma^2
%\end{pmatrix}=
\begin{pmatrix}
\sigma^2_{b,A} + \sigma^2  &  \rho\sigma_{b,A}\sigma_{b,B}\\
\rho\sigma_{b,A}\sigma_{b,B} & \sigma^2_{b,B}+\sigma^2  \\       
\end{pmatrix}
\end{equation}

And that's equal to (see VarCorr output above):


\begin{equation}
\begin{pmatrix}
5.93634+0.07266  & \rho\sigma_{b,A}\sigma_{b,B}=0.1645525\\
 0.1645525 & 0.0045617+ 0.07266\\
\end{pmatrix}
\end{equation}


Note that $\hat{\rho}$ is shown in VarCorr output (at the bottom) and can be computed since $\frac{Covar}{\sigma_a\times \sigma_b}=\rho$ and we know all the quantities on the LHS:

\begin{Schunk}
\begin{Sinput}
> 0.1645525/(sqrt(5.93634)*sqrt(0.0045617))
\end{Sinput}
\begin{Soutput}
[1] 0.99996
\end{Soutput}
\end{Schunk}

How to recover, from V, the correlation of 1 in the lmer random effects output of fm2? Is that 1 supposed to represent 0.99996?

\subsection{No varying intercepts, only slopes for each level}

\begin{Schunk}
\begin{Sinput}
> fm3<-lmer(wear~material-1 + (material-1|Subject),BHHshoes)
> ranef(fm3)
\end{Sinput}
\begin{Soutput}
$Subject
   materialA materialB
1    2.71318   2.78838
2   -2.28776  -2.35117
3    0.21003   0.21585
4    3.34435   3.43705
5    0.41145   0.42286
6   -4.25374  -4.37165
7   -1.16241  -1.19463
8    0.21137   0.21723
9   -1.74925  -1.79774
10   2.56278   2.63382
\end{Soutput}
\end{Schunk}

The model is

\begin{equation}
Y_{ijk} = \beta_j + b_{ij} + \epsilon_{ijk} 
\end{equation}

The random effects are:

$b_{ij}=\begin{pmatrix}
b_{i1}\\
b_{i12}
\end{pmatrix}
\sim N(0,\sigma_b^2)$, where $\sigma_b^2=
\begin{pmatrix}
\sigma_1^2 & \rho\sigma_1 \sigma_2 \\
\rho\sigma_1 \sigma_2 & \sigma_2^2 \\
\end{pmatrix}$. 

We can recover these values from:

\begin{Schunk}
\begin{Sinput}
> var(ranef(fm3)$Subject)
\end{Sinput}
\begin{Soutput}
          materialA materialB
materialA    5.9012    6.0648
materialB    6.0648    6.2329
\end{Soutput}
\end{Schunk}

$\hat{\rho}$ is 1 because $1\times 
\sqrt{5.9012}*\sqrt{6.2329}=6.0648$.


Here, V is

\begin{equation}
V =
%\begin{pmatrix}
%\sigma_b^2 + \sigma^2 & \sigma_b^2\\
%\sigma_b^2 & \sigma_b^2 + \sigma^2
%\end{pmatrix}=
\begin{pmatrix}
\sigma^2_{b,A} + \sigma^2  &  \rho\sigma_{b,A}\sigma_{b,B}\\
\rho\sigma_{b,A}\sigma_{b,B} & \sigma^2_{b,B}+\sigma^2  \\       
\end{pmatrix}
\end{equation}

\textbf{Note that the interpretation of the random effects is different from fm2: here, a random effect is computed for each material separately.}


From the VarCorr output, we have $\hat{V}$:

\begin{equation}
\begin{pmatrix}
5.9363 + 0.26956^2  &  1 \times 2.4365\times 2.5040 \\
1 \times 2.4365\times 2.5040 & 6.27+0.26956^2  \\       
\end{pmatrix}
\end{equation}


One insight is that V can be derived from the random effects variance components, and the error term's variance component:

\begin{equation}
V=
\begin{pmatrix}
\sigma^2_{b,A} &\rho\sigma_{b,A}\sigma_{b,B}\\
\rho\sigma_{b,A}\sigma_{b,B} & \sigma^2_{b,B}\\
\end{pmatrix}
+
\begin{pmatrix}
\sigma^2 & 0\\
0 & \sigma^2\\
\end{pmatrix}
\end{equation}

\subsection{Nested models (e.g., Worker/Machine)}

The model is:

\begin{equation}
Y_{ijk} = \beta_j + b_i + b_{ij} + \epsilon_{ijk}
\end{equation}

Here, we force force all random effects to be independent.
Observations between workers are independent, but observations on the same worker are correlated.

$b_i \sim N(0,\sigma_1^2), b_{ij} \sim N(0,\sigma_2^2)$, and $\epsilon\sim N(0,\sigma^2)$. $i$ is Worker, $j$ is machine, and $k$ is replicate.  

\begin{Schunk}
\begin{Sinput}
> fm1<-lmer(score~Machine-1+(1|Worker/Machine),
   data=Machines)
\end{Sinput}
\end{Schunk}

The variance components in fm1:

\small
\begin{tabular}{rlll}
  \hline
Comp.\ & Groups & Name & Var\\ 
  \hline
$\hat{\sigma}_2^2$ & Machine:Worker & (Int) & 13.909  \\ 
$\hat{\sigma}_1^2$  & Worker & (Int) & 22.858  \\ 
$\hat{\sigma}^2$& Res &  &  0.925 \\ 
   \hline
\end{tabular}

Number of obs: 54, groups: Machine:Worker, 18; Worker, 6.

\normalsize

For observations on Worker $i$, 

\begin{equation}
Var(Y_{ijk})= \sigma_1^2 + \sigma_2^2 + \sigma^2 
\end{equation}

Variance between machines within workers:
\begin{equation}
Covar(Y_{ijk},Y_{ijk'})= \sigma_1^2 + \sigma_2^2
\end{equation}

Variance between workers:
\begin{equation}
Covar(Y_{ijk},Y_{ij'k'})= \sigma_1^2
\end{equation}

Note:

1. $\hat{\sigma}_1^2$ all observations have the same variance;

2. $\hat{\sigma}_2^2$: the covariance between observations corresponding to the same worker using different machines is the same, for any pair of machines.


\begin{verbatim}
> ranef(fm1)
$`Machine:Worker`    $Worker
    (Intercept)
A:6     1.91609     6 -7.514666
A:2     1.55253     2 -1.375925
\end{verbatim}

In this model, the sum of the random effects for Worker 1 on Machine A is

$s_1 = b_1 + b_{11}$

\begin{verbatim}
> ranef(fm1)
...
     $`Machine:Worker`      $Worker
        (Intercept)
s1 = A:1    -0.75012 +      1.044598 = 0.29448
\end{verbatim}

and for Worker 1 on machine B,

$s_2 = b_1 + b_{21}$.

\begin{verbatim}
> ranef(fm1)
...
    $`Machine:Worker`      $Worker
     (Intercept)
s2 = B:1     1.50002 +      1.044598 =  2.5446
\end{verbatim}

For all Workers and machines, we can obtain these random effects $s$ from this matrix:

\begin{Schunk}
\begin{Sinput}
> mat<-matrix(
     unlist(ranef(fm1)$`Machine:Worker`),6,3) 
> + 
     matrix(unlist(ranef(fm1)$Worker),6,3)
\end{Sinput}
\begin{Soutput}
          [,1]      [,2]      [,3]
[1,] -7.514666 -7.514666 -7.514666
[2,] -1.375925 -1.375925 -1.375925
[3,] -0.059823 -0.059823 -0.059823
[4,]  1.044598  1.044598  1.044598
[5,]  5.361045  5.361045  5.361045
[6,]  2.544771  2.544771  2.544771
\end{Soutput}
\end{Schunk}

\begin{Schunk}
\begin{Sinput}
> mat
\end{Sinput}
\begin{Soutput}
         A        B        C
6  1.91609 -8.97590  2.48677
2  1.55253  0.60682 -2.99667
4 -1.03937  2.41736 -1.41440
1 -0.75012  1.50002 -0.11421
3  1.77775  2.29952 -0.81481
5 -3.45687  2.15218  2.85331
\end{Soutput}
\end{Schunk}

Using lmer, we have $b_{i}$ and $b_{ij}$ independent, but $s_1$ and $s_2$ are
correlated via the common term $b_1$. We can recover the correlations between machine through the vcov matrix of the random effects (BLUPs) (\textbf{but note that we never see this in the lmer output---what's the significance of the fact that these are correlated?}):

\begin{Schunk}
\begin{Sinput}
> var(mat)
\end{Sinput}
\begin{Soutput}
        A       B       C
A  4.5670 -4.6492 -1.9288
B -4.6492 19.7897 -4.6925
C -1.9288 -4.6925  5.1966
\end{Soutput}
\end{Schunk}





\subsection{Varying intercepts and slopes (no correlation)}

\begin{equation}
Y_{ijk} = \beta_j + b_{ij} + \epsilon_{ijk}
\end{equation}



\begin{Schunk}
\begin{Sinput}
> fm3<-lmer(score~Machine-1+
               (Machine-1|Worker),
             data=Machines)
\end{Sinput}
\end{Schunk}

\begin{Schunk}
\begin{Sinput}
> ranef(fm3)
\end{Sinput}
\begin{Soutput}
$Worker
  MachineA  MachineB MachineC
6 -5.59160 -16.58381  -5.0305
2  0.18387  -0.80332  -4.2823
4 -1.02388   2.32846  -1.4144
1  0.31199   2.55323   0.9304
3  6.96922   7.77935   4.4733
5 -0.84961   4.72610   5.3235
\end{Soutput}
\end{Schunk}

The random effects for Worker 1 on Machine A is

$s_1 = b_{11}=0.31199$

and for Worker 1 on Machine B,

$s_2 = b_{12}=2.55323$.

The `Machine independent' Worker random effect (varying intercept) $b_i$ has been dropped. 
We have $b_{11}$ correlated with $b_{12}$. We can see this when we recover the (co-)variances between machines from the random effects: 

\small
\begin{Schunk}
\begin{Sinput}
> var(ranef(fm3)$Worker)
\end{Sinput}
\begin{Soutput}
         MachineA MachineB
MachineA   16.347   28.239
MachineB   28.239   74.093
MachineC   11.146   29.181
         MachineC
MachineA   11.146
MachineB   29.181
MachineC   18.972
\end{Soutput}
\end{Schunk}
\normalsize

Also, the variances for each machine (16, 74, 18)
are also allowed to be different. Here are the variance components:

\tiny
\begin{tabular}{rlllll}
  \hline
Comp.\ & Groups & Name & Variance  & Corr$_{1,\cdot}$ &  Corr$_{2,\cdot}$  \\ 
  \hline
$\hat{\sigma}_{j=1}^2$ & Worker & A & 16.640   &    \\ 
$\hat{\sigma}_{j=2}^2$ &  & B & 74.395 & 
$\hat{\rho}_{1,2}$ 0.803 &    \\ 
$\hat{\sigma}_{j=3}^2$ &  & C & 19.268 &
$\hat{\rho}_{1,3}$ 0.623 & $\hat{\rho}_{2,3}$ 0.771   \\ 
$\hat{\sigma}^2$ & Res &  &  0.925 &  &    \\ 
   \hline
\end{tabular}
\normalsize


\begin{equation}
Var(Y_{ijk})= \sigma_j^2 + \sigma^2
\end{equation}

\begin{equation}
Covar(Y_{ijk},Y_{ijk'})= \sigma_j^2
\end{equation}

\begin{equation}
Covar(Y_{ijk},Y_{ij'k'})= \rho_{j,j'} \sigma_j\sigma_{j'}
\end{equation}



Note that the BLUPs' vcov matrix reflects the estimated values:

\begin{Schunk}
\begin{Sinput}
> diag(var(ranef(fm3)$Worker))
\end{Sinput}
\begin{Soutput}
MachineA MachineB MachineC 
  16.347   74.093   18.972 
\end{Soutput}
\begin{Sinput}
> cor(ranef(fm3)$Worker)
\end{Sinput}
\begin{Soutput}
         MachineA MachineB
MachineA  1.00000  0.81141
MachineB  0.81141  1.00000
MachineC  0.63292  0.77832
         MachineC
MachineA  0.63292
MachineB  0.77832
MachineC  1.00000
\end{Soutput}
\begin{Sinput}
> # look at the fm3 output 
> ## (the random effects table)
\end{Sinput}
\end{Schunk}


1. $\hat{\sigma}_j^2$ the variance of an observation depends on the machine being used; 

2. $\rho_{j,j'} \sigma_j\sigma_{j'}$ the covariance between observations corresponding to the same worker using different
machines is different, for different pairs of machines.

\begin{verbatim}
> var(ranef(fm3)$Worker)
         MachineA MachineB MachineC
MachineA   16.347   28.239   11.146
MachineB   28.239   74.093   29.181
MachineC   11.146   29.181   18.972
\end{verbatim}

\begin{equation}
\begin{pmatrix}
\sigma_{A}^2 & Cov_{A,B}     & Cov_{A,C}\\  
               & \sigma_{B}^2 & Cov_{B,C} \\
              &               & \sigma_{C}^2\\
\end{pmatrix}
\end{equation}

Note that, for given machines $j$ and $j'$, say A, B: 

$Covar(Y_{ijk},Y_{ij'k'}) = Cov_{A,B}=28.239 \approx  \rho_{A,B} \sigma_{A} \sigma_{B}
= .803 \times \sqrt{16.347} \times \sqrt{74.093} = 27.946$.  

\subsection{Comparing fm1 and fm3}

The sum of fm1's (Worker/Machine) ranefs ($b_{ij}+b_i$) are roughly the same as fm3's (Machine-1$\mid$ Worker) random effects $b_{ij}$ for each machine. \textbf{In other words, the random effect $b_i$ is folded into $b_{ij}$ in fm3.}

\begin{Schunk}
\begin{Sinput}
> #fm1's ranefs summed up are 
> ## roughly the same as the fm3 ranefs:
> matrix(unlist(ranef(fm1)$`Machine:Worker`),6,3) +
   matrix(unlist(ranef(fm1)$Worker),6,3)
\end{Sinput}
\begin{Soutput}
         [,1]      [,2]     [,3]
[1,] -5.59858 -16.49057 -5.02789
[2,]  0.17661  -0.76911 -4.37259
[3,] -1.09920   2.35754 -1.47422
[4,]  0.29448   2.54462  0.93039
[5,]  7.13879   7.66056  4.54624
[6,] -0.91210   4.69695  5.39808
\end{Soutput}
\begin{Sinput}
> ranef(fm3)
\end{Sinput}
\begin{Soutput}
$Worker
  MachineA  MachineB MachineC
6 -5.59160 -16.58381  -5.0305
2  0.18387  -0.80332  -4.2823
4 -1.02388   2.32846  -1.4144
1  0.31199   2.55323   0.9304
3  6.96922   7.77935   4.4733
5 -0.84961   4.72610   5.3235
\end{Soutput}
\end{Schunk}

\section{How the random effects are 'predicted' when using the ranef() command (section 4.4.3).}

In linear mixed models, we fit models like these (the Ware-Laird formulation--see Pinheiro and Bates 2000, for example):

\begin{equation} 
Y = X\beta + Zu + \epsilon
\end{equation}

Let $u\sim N(0,\sigma_u^2)$, and this is independent from $\epsilon\sim N(0,\sigma^2)$.  

Given $Y$, the ``minimum mean square error predictor'' of $u$ is the conditional expectation:

\begin{equation}
\hat{u} = E(u\mid Y)
\end{equation}

We can find $E(u\mid Y)$ as follows. We write the joint distribution of $Y$ and $u$ as:

\begin{equation}
\begin{pmatrix}
Y \\
u
\end{pmatrix}
= 
N\left(
\begin{pmatrix}
X\beta\\
0
\end{pmatrix},
\begin{pmatrix}
V_Y & C_{Y,u}\\
C_{u,Y} & V_u \\
\end{pmatrix}
\right)
\end{equation}

$V_Y, C_{Y,u}, C_{u,Y}, V_u$ are the various variance-covariance matrices. 
It is a fact (need to track this down) that

\begin{equation}
u\mid Y \sim N(C_{u,Y}V_Y^{-1}(Y-X\beta)), 
Y_u - C_{u,Y} V_Y^{-1} C_{Y,u})
\end{equation}

This apparently allows you to derive the BLUPs:

\begin{equation}
\hat{u}= C_{u,Y}V_Y^{-1}(Y-X\beta))
\end{equation}

Substituting $\hat{\beta}$ for $\beta$, we get:

\begin{equation}
BLUP(u)= \hat{u}(\hat{\beta})=C_{u,Y}V_Y^{-1}(Y-X\hat{\beta}))
\end{equation}

Here's an example with R:

\begin{Schunk}
\begin{Sinput}
> # Calculate the predicted random effects by hand for the ergoStool data
> (fm1<-lmer(effort~Type-1 + (1|Subject),ergoStool))
\end{Sinput}
\begin{Soutput}
Linear mixed model fit by REML 
Formula: effort ~ Type - 1 + (1 | Subject) 
   Data: ergoStool 
 AIC BIC logLik deviance REMLdev
 133 143  -60.6      122     121
Random effects:
 Groups   Name        Variance
 Subject  (Intercept) 1.78    
 Residual             1.21    
 Std.Dev.
 1.33    
 1.10    
Number of obs: 36, groups: Subject, 9

Fixed effects:
       Estimate Std. Error t value
TypeT1    8.556      0.576    14.8
TypeT2   12.444      0.576    21.6
TypeT3   10.778      0.576    18.7
TypeT4    9.222      0.576    16.0

Correlation of Fixed Effects:
       TypeT1 TypeT2 TypeT3
TypeT2 0.595               
TypeT3 0.595  0.595        
TypeT4 0.595  0.595  0.595 
\end{Soutput}
\begin{Sinput}
> ## Here are the BLUPs we will estimate by hand:
> ranef(fm1)
\end{Sinput}
\begin{Soutput}
$Subject
  (Intercept)
1  1.7088e+00
2  1.7088e+00
3  4.2720e-01
4 -8.5439e-01
5 -1.4952e+00
6 -1.3546e-14
7  4.2720e-01
8 -1.7088e+00
9 -2.1360e-01
\end{Soutput}
\begin{Sinput}
> ## this gives us all the variance components:
> VarCorr(fm1)
\end{Sinput}
\begin{Soutput}
$Subject
            (Intercept)
(Intercept)      1.7755
attr(,"stddev")
(Intercept) 
     1.3325 
attr(,"correlation")
            (Intercept)
(Intercept)           1

attr(,"sc")
[1] 1.1003
\end{Soutput}
\begin{Sinput}
> # First, calculate the predicted random effect for subject 1:
> 
> ## The variance for the random effect subject is the term C_{u,Y}:
> covar.u.y<-VarCorr(fm1)$Subject[1]
> # Estimated covariance between u_1 and Y_1
> ## make up a var-covar matrix from this:
> (cov.u.Y<-matrix(covar.u.y,1,4))
\end{Sinput}
\begin{Soutput}
       [,1]   [,2]   [,3]   [,4]
[1,] 1.7755 1.7755 1.7755 1.7755
\end{Soutput}
\begin{Sinput}
> # Estimated variance matrix for Y_1
> (V.Y<-matrix(1.7755,4,4)+diag(1.2106,4,4))
\end{Sinput}
\begin{Soutput}
       [,1]   [,2]   [,3]   [,4]
[1,] 2.9861 1.7755 1.7755 1.7755
[2,] 1.7755 2.9861 1.7755 1.7755
[3,] 1.7755 1.7755 2.9861 1.7755
[4,] 1.7755 1.7755 1.7755 2.9861
\end{Soutput}
\begin{Sinput}
> # Extract observations for subject 1
> (Y<-matrix(ergoStool$effort[1:4],4,1))
\end{Sinput}
\begin{Soutput}
     [,1]
[1,]   12
[2,]   15
[3,]   12
[4,]   10
\end{Soutput}
\begin{Sinput}
> # Estimated fixed effects
> (beta.hat<-matrix(fixef(fm1),4,1))
\end{Sinput}
\begin{Soutput}
        [,1]
[1,]  8.5556
[2,] 12.4444
[3,] 10.7778
[4,]  9.2222
\end{Soutput}
\begin{Sinput}
> # Predicted random effect
> cov.u.Y %*% solve(V.Y)%*%(Y-beta.hat)
\end{Sinput}
\begin{Soutput}
       [,1]
[1,] 1.7087
\end{Soutput}
\begin{Sinput}
> # Compare with ranef command
> ranef(fm1)$Subject[1,1]
\end{Sinput}
\begin{Soutput}
[1] 1.7088
\end{Soutput}
\begin{Sinput}
> # Calculate predicted random effects for all subjects
> t(cov.u.Y %*% solve(V.Y)%*%(matrix(ergoStool$effort,4,9)-matrix(fixef(fm1),4,9)))
\end{Sinput}
\begin{Soutput}
             [,1]
 [1,]  1.7087e+00
 [2,]  1.7087e+00
 [3,]  4.2717e-01
 [4,] -8.5435e-01
 [5,] -1.4951e+00
 [6,] -1.3906e-14
 [7,]  4.2717e-01
 [8,] -1.7087e+00
 [9,] -2.1359e-01
\end{Soutput}
\begin{Sinput}
> ranef(fm1)
\end{Sinput}
\begin{Soutput}
$Subject
  (Intercept)
1  1.7088e+00
2  1.7088e+00
3  4.2720e-01
4 -8.5439e-01
5 -1.4952e+00
6 -1.3546e-14
7  4.2720e-01
8 -1.7088e+00
9 -2.1360e-01
\end{Soutput}
\end{Schunk}

\section{Correlations of fixed effects}

For an ordinary linear model, the covariance matrix (from which we can get the correlation matrix) of $\hat{beta}$ is

\begin{equation}
\sigma^2 \times (X^T X)^{-1}.
\end{equation}

For a mixed effects model, the standard deviations (standard errors) and correlations for the fixed effects estimators are listed at the end of the lmer output. 

\begin{Schunk}
\begin{Sinput}
> lm.full<-lmer(wear~material-1+(1|Subject), data = BHHshoes)
\end{Sinput}
\end{Schunk}

The estimated correlation between $\hat{beta}_1$ and $\hat{beta}_2$ is $0.988$.
In this case, we have simple forms for the parameter estimators:

\begin{equation}
\hat{\beta}_1 = (Y_{1,1} + Y_{2,1} + \dots + Y_{10,1})/10
\end{equation}


\begin{equation}
\hat{\beta}_2 = (Y_{1,2} + Y_{2,2} + \dots + Y_{10,2})/10
\end{equation}

\begin{Schunk}
\begin{Sinput}
> b1.vals<-subset(BHHshoes,material=="A")$wear
> b2.vals<-subset(BHHshoes,material=="B")$wear
> vcovmatrix<-var(cbind(b1.vals,b2.vals))
> covar<-vcovmatrix[1,2]
> sds<-sqrt(diag(vcovmatrix))
> covar/(sds[1]*sds[2])
\end{Sinput}
\begin{Soutput}
b1.vals 
0.98823 
\end{Soutput}
\begin{Sinput}
> #cf:
> covar/((0.786*sqrt(10))^2)  
\end{Sinput}
\begin{Soutput}
[1] 0.98752
\end{Soutput}
\end{Schunk}

In a regular linear model version, we would have had:

\begin{Schunk}
\begin{Sinput}
> fm.lm<-lm(wear~material-1,BHHshoes)
> X<-model.matrix(fm.lm)
> 2.49^2*solve(t(X)%*%X)
\end{Sinput}
\begin{Soutput}
          materialA materialB
materialA   0.62001   0.00000
materialB   0.00000   0.62001
\end{Soutput}
\end{Schunk}

because $Var(\hat{\beta}) = \hat{\sigma}^2 (X^T X)^{-1}$.

From this, see if you can work out the covariance, and where the estimated correlation comes from, using the remainder of the lmer output above.

\begin{Schunk}
\begin{Sinput}
> b1.diffs<-b1.vals-mean(b1.vals)
> b2.diffs<-b2.vals-mean(b2.vals)
> b1.diffs<-b1.vals-mean(BHHshoes$wear)
> b2.diffs<-b2.vals-mean(BHHshoes$wear)
> covar<-t(b1.diffs)%*%b2.diffs
> b1.sd<-sd(b1.vals)
> b2.sd<-sd(b2.vals)
> corr<-covar/(b1.sd*b2.sd)
\end{Sinput}
\end{Schunk}

How does this work for multiple factors?

\begin{Schunk}
\begin{Sinput}
> m1<-lmer(effort~Type-1+(1|Subject),ergoStool)
> T1.vals<-subset(ergoStool,Type=="T1")$effort
> T2.vals<-subset(ergoStool,Type=="T2")$effort
> T3.vals<-subset(ergoStool,Type=="T3")$effort
> T4.vals<-subset(ergoStool,Type=="T4")$effort
> vals<-cbind(T1.vals,T2.vals,T3.vals,T4.vals)
> ## compute variance covariance matrix:
> vcovmat<-var(vals)
> ## get sd's of each level:
> sds<-sqrt(diag(vcovmat))
> ## T1.T2 correlation, the sds come from the model fit:
> 1.7222/(1.728*1.728)
\end{Sinput}
\begin{Soutput}
[1] 0.57676
\end{Soutput}
\end{Schunk}

Note: Not sure if the above is correct (the case of multiple levels in a factor).

\section{$\sigma_b^2$ describes both between-block variance, and within block covariance}

Consider the following model:

\begin{equation}
Y_{ij} = b_i + e_{ij},
\end{equation}


with $b_i\sim N(0,\sigma^2_b)$, $e_{ij}~N(0,\sigma^2)$.

Now try this in R (corresponding to $\sigma=1, \sigma_b=100, i=1,2,3$ and $j=1,2,3$):

\begin{Schunk}
\begin{Sinput}
> block<-gl(3,3)
> ## very small within group:
> eij<-rnorm(9,0,1)
> ## very high between group variance:
> ei<-rnorm(3,0,100)
> y<-rep(ei,each=3)+eij
> plot(block,y)
> fm1<-lm(y~1)
> aggregated<-tapply(y,block,mean)
> agg.data<-data.frame(means=aggregated,block=factor(1:3))
> fm1a<-lm(y~1,agg.data)
> fm3<-lmer(y~1+(1|block))
> a<-y[c(1,4,7)]
> b<-y[c(1,4,7)+1]
> c<-y[c(1,4,7)+2]
> (cov(a,b)+cov(a,c)+cov(b,c))/3
\end{Sinput}
\begin{Soutput}
[1] 5664.2
\end{Soutput}
\begin{Sinput}
> ## more like what we experience:
> block<-gl(3,3)
> ## large within group:
> eij<-rnorm(9,0,100)
> ## small between group:
> ei<-rnorm(3,0,1)
> y<-rep(ei,each=3)+eij
> plot(block,y)
> fm1<-lm(y~1)
> aggregated<-tapply(y,block,mean)
> agg.data<-data.frame(means=aggregated,block=factor(1:3))
> fm1a<-lm(y~1,agg.data)
> fm3<-lmer(y~1+(1|block))
\end{Sinput}
\end{Schunk}

Perhaps it's just worth remembering that a variance is a covariance of a random variable with itself, and then consider the model formulation. If we have

\begin{equation}
Y_{ij} = \mu + b_i + \epsilon_{ij}
\end{equation}

where i is the group, j is the replication, if we \textit{define} $b_i\sim N(0, \sigma^2_b)$, and refer to $\sigma^2_b$ as the between group variance, then we must have


\begin{equation}
\begin{split}
Cov(Y_{i1}, Y_{i2}) =& Cov(\mu + b_i + \epsilon_{i1} , \mu + b_i + \epsilon_{i2})\\
=& \explain{Cov(\mu, \mu)}{=0} + 
 \explain{Cov(\mu, b_i)}{=0} +
 \explain{Cov(\mu, \epsilon_{i2})}{=0} +
 \explain{Cov(b_i,\mu)}{=0} +
  \explain{Cov(b_i,b_i)}{+ve} \dots\\
  =&  Cov(b_i, b_i) = Var(b_i) = \sigma^2_b\\
\end{split}
\end{equation}


\bibliographystyle{plain}
\bibliography{/Users/shravanvasishth/Dropbox/Bibliography/bibcleaned}

Cheat sheet template taken from Winston Chang: http://www.stdout.org/$\sim$winston/latex/



\end{multicols}
\end{document}


\subsection{Common \texttt{documentclass} options}
\newlength{\MyLen}
\settowidth{\MyLen}{\texttt{letterpaper}/\texttt{a4paper} \ }
\begin{tabular}{@{}p{\the\MyLen}%
                @{}p{\linewidth-\the\MyLen}@{}}
\texttt{10pt}/\texttt{11pt}/\texttt{12pt} & Font size. \\
\texttt{letterpaper}/\texttt{a4paper} & Paper size. \\
\texttt{twocolumn} & Use two columns. \\
\texttt{twoside}   & Set margins for two-sided. \\
\texttt{landscape} & Landscape orientation.  Must use
                     \texttt{dvips -t landscape}. \\
\texttt{draft}     & Double-space lines.
\end{tabular}

Usage:
\verb!\documentclass[!\textit{opt,opt}\verb!]{!\textit{class}\verb!}!.


\subsection{Packages}
\settowidth{\MyLen}{\texttt{multicol} }
\begin{tabular}{@{}p{\the\MyLen}%
                @{}p{\linewidth-\the\MyLen}@{}}
%\begin{tabular}{@{}ll@{}}
\texttt{fullpage}  &  Use 1 inch margins. \\
\texttt{anysize}   &  Set margins: \verb!\marginsize{!\textit{l}%
                        \verb!}{!\textit{r}\verb!}{!\textit{t}%
                        \verb!}{!\textit{b}\verb!}!.            \\
\texttt{multicol}  &  Use $n$ columns: 
                        \verb!\begin{multicols}{!$n$\verb!}!.   \\
\texttt{latexsym}  &  Use \LaTeX\ symbol font. \\
\texttt{graphicx}  &  Show image:
                        \verb!\includegraphics[width=!%
                        \textit{x}\verb!]{!%
                        \textit{file}\verb!}!. \\
\texttt{url}       & Insert URL: \verb!\url{!%
                        \textit{http://\ldots}%
                        \verb!}!.
\end{tabular}

Use before \verb!\begin{document}!. 
Usage: \verb!\usepackage{!\textit{package}\verb!}!


\subsection{Title}
\settowidth{\MyLen}{\texttt{.author.text.} }
\begin{tabular}{@{}p{\the\MyLen}%
                @{}p{\linewidth-\the\MyLen}@{}}
\verb!\author{!\textit{text}\verb!}! & Author of document. \\
\verb!\title{!\textit{text}\verb!}!  & Title of document. \\
\verb!\date{!\textit{text}\verb!}!   & Date. \\
\end{tabular}

These commands go before \verb!\begin{document}!.  The declaration
\verb!\maketitle! goes at the top of the document.

\subsection{Miscellaneous}
\settowidth{\MyLen}{\texttt{.pagestyle.empty.} }
\begin{tabular}{@{}p{\the\MyLen}%
                @{}p{\linewidth-\the\MyLen}@{}}
\verb!\pagestyle{empty}!     &   Empty header, footer
                                 and no page numbers. \\
\verb!\tableofcontents!      &   Add a table of contents here. \\

\end{tabular}



\section{Document structure}
\begin{multicols}{2}
\verb!\part{!\textit{title}\verb!}!  \\
\verb!\chapter{!\textit{title}\verb!}!  \\
\verb!\section{!\textit{title}\verb!}!  \\
\verb!\subsection{!\textit{title}\verb!}!  \\
\verb!\subsubsection{!\textit{title}\verb!}!  \\
\verb!\paragraph{!\textit{title}\verb!}!  \\
\verb!\subparagraph{!\textit{title}\verb!}!
\end{multicols}
{\raggedright
Use \verb!\setcounter{secnumdepth}{!$x$\verb!}! suppresses heading
numbers of depth $>x$, where \verb!chapter! has depth 0.
Use a \texttt{*}, as in \verb!\section*{!\textit{title}\verb!}!,
to not number a particular item---these items will also not appear
in the table of contents.
}

\subsection{Text environments}
\settowidth{\MyLen}{\texttt{.begin.quotation.}}
\begin{tabular}{@{}p{\the\MyLen}%
                @{}p{\linewidth-\the\MyLen}@{}}
\verb!\begin{comment}!    &  Comment (not printed). Requires \texttt{verbatim} package. \\
\verb!\begin{quote}!      &  Indented quotation block. \\
\verb!\begin{quotation}!  &  Like \texttt{quote} with indented paragraphs. \\
\verb!\begin{verse}!      &  Quotation block for verse.
\end{tabular}

\subsection{Lists}
\settowidth{\MyLen}{\texttt{.begin.description.}}
\begin{tabular}{@{}p{\the\MyLen}%
                @{}p{\linewidth-\the\MyLen}@{}}
\verb!\begin{enumerate}!        &  Numbered list. \\
\verb!\begin{itemize}!          &  Bulleted list. \\
\verb!\begin{description}!      &  Description list. \\
\verb!\item! \textit{text}      &  Add an item. \\
\verb!\item[!\textit{x}\verb!]! \textit{text}
                                &  Use \textit{x} instead of normal
                        bullet or number.  Required for descriptions. \\
\end{tabular}




\subsection{References}
\settowidth{\MyLen}{\texttt{.pageref.marker..}}
\begin{tabular}{@{}p{\the\MyLen}%
                @{}p{\linewidth-\the\MyLen}@{}}
\verb!\label{!\textit{marker}\verb!}!   &  Set a marker for cross-reference, 
                          often of the form \verb!\label{sec:item}!. \\
\verb!\ref{!\textit{marker}\verb!}!   &  Give section/body number of marker. \\
\verb!\pageref{!\textit{marker}\verb!}! &  Give page number of marker. \\
\verb!\footnote{!\textit{text}\verb!}!  &  Print footnote at bottom of page. \\
\end{tabular}


\subsection{Floating bodies}
\settowidth{\MyLen}{\texttt{.begin.equation..place.}}
\begin{tabular}{@{}p{\the\MyLen}%
                @{}p{\linewidth-\the\MyLen}@{}}
\verb!\begin{table}[!\textit{place}\verb!]!     &  Add numbered table. \\
\verb!\begin{figure}[!\textit{place}\verb!]!    &  Add numbered figure. \\
\verb!\begin{equation}[!\textit{place}\verb!]!  &  Add numbered equation. \\
\verb!\caption{!\textit{text}\verb!}!           &  Caption for the body. \\
\end{tabular}

The \textit{place} is a list valid placements for the body.  \texttt{t}=top,
\texttt{h}=here, \texttt{b}=bottom, \texttt{p}=separate page, \texttt{!}=place even if ugly.  Captions and label markers should be within the environment.

%---------------------------------------------------------------------------

\section{Text properties}

\subsection{Font face}
\newcommand{\FontCmd}[3]{\PBS\verb!\#1{!\textit{text}\verb!}!  \> %
                         \verb!{\#2 !\textit{text}\verb!}! \> %
                         \#1{#3}}
\begin{tabular}{@{}l@{}l@{}l@{}}
\textit{Command} & \textit{Declaration} & \textit{Effect} \\
\verb!\textrm{!\textit{text}\verb!}!                    & %
        \verb!{\rmfamily !\textit{text}\verb!}!               & %
        \textrm{Roman family} \\
\verb!\textsf{!\textit{text}\verb!}!                    & %
        \verb!{\sffamily !\textit{text}\verb!}!               & %
        \textsf{Sans serif family} \\
\verb!\texttt{!\textit{text}\verb!}!                    & %
        \verb!{\ttfamily !\textit{text}\verb!}!               & %
        \texttt{Typewriter family} \\
\verb!\textmd{!\textit{text}\verb!}!                    & %
        \verb!{\mdseries !\textit{text}\verb!}!               & %
        \textmd{Medium series} \\
\verb!\textbf{!\textit{text}\verb!}!                    & %
        \verb!{\bfseries !\textit{text}\verb!}!               & %
        \textbf{Bold series} \\
\verb!\textup{!\textit{text}\verb!}!                    & %
        \verb!{\upshape !\textit{text}\verb!}!               & %
        \textup{Upright shape} \\
\verb!\textit{!\textit{text}\verb!}!                    & %
        \verb!{\itshape !\textit{text}\verb!}!               & %
        \textit{Italic shape} \\
\verb!\textsl{!\textit{text}\verb!}!                    & %
        \verb!{\slshape !\textit{text}\verb!}!               & %
        \textsl{Slanted shape} \\
\verb!\textsc{!\textit{text}\verb!}!                    & %
        \verb!{\scshape !\textit{text}\verb!}!               & %
        \textsc{Small Caps shape} \\
\verb!\emph{!\textit{text}\verb!}!                      & %
        \verb!{\em !\textit{text}\verb!}!               & %
        \emph{Emphasized} \\
\verb!\textnormal{!\textit{text}\verb!}!                & %
        \verb!{\normalfont !\textit{text}\verb!}!       & %
        \textnormal{Document font} \\
\verb!\underline{!\textit{text}\verb!}!                 & %
                                                        & %
        \underline{Underline}
\end{tabular}

The command (t\textit{tt}t) form handles spacing better than the
declaration (t{\itshape tt}t) form.

\subsection{Font size}
\setlength{\columnsep}{14pt} % Need to move columns apart a little
\begin{multicols}{2}
\begin{tabbing}
\verb!\footnotesize!          \= \kill
\verb!\tiny!                  \>  \tiny{tiny} \\
\verb!\scriptsize!            \>  \scriptsize{scriptsize} \\
\verb!\footnotesize!          \>  \footnotesize{footnotesize} \\
\verb!\small!                 \>  \small{small} \\
\verb!\normalsize!            \>  \normalsize{normalsize} \\
\verb!\large!                 \>  \large{large} \\
\verb!\Large!                 \=  \Large{Large} \\  % Tab hack for new column
\verb!\LARGE!                 \>  \LARGE{LARGE} \\
\verb!\huge!                  \>  \huge{huge} \\
\verb!\Huge!                  \>  \Huge{Huge}
\end{tabbing}
\end{multicols}
\setlength{\columnsep}{1pt} % Set column separation back

These are declarations and should be used in the form
\verb!{\small! \ldots\verb!}!, or without braces to affect the entire
document.


\subsection{Verbatim text}

\settowidth{\MyLen}{\texttt{.begin.verbatim..} }
\begin{tabular}{@{}p{\the\MyLen}%
                @{}p{\linewidth-\the\MyLen}@{}}
\verb@\begin{verbatim}@ & Verbatim environment. \\
\verb@\begin{verbatim*}@ & Spaces are shown as \verb*@ @. \\
\verb@\verb!text!@ & Text between the delimiting characters (in this case %
                      `\texttt{!}') is verbatim.
\end{tabular}


\subsection{Justification}
\begin{tabular}{@{}ll@{}}
\textit{Environment}  &  \textit{Declaration}  \\
\verb!\begin{center}!      & \verb!\centering!  \\
\verb!\begin{flushleft}!  & \verb!\raggedright! \\
\verb!\begin{flushright}! & \verb!\raggedleft!  \\
\end{tabular}

\subsection{Miscellaneous}
\verb!\linespread{!$x$\verb!}! changes the line spacing by the
multiplier $x$.





\section{Text-mode symbols}

\subsection{Symbols}
\begin{tabular}{@{}l@{\hspace{1em}}l@{\hspace{2em}}l@{\hspace{1em}}l@{\hspace{2em}}l@{\hspace{1em}}l@{\hspace{2em}}l@{\hspace{1em}}l@{}}
\&              &  \verb!\&! &
\_              &  \verb!\_! &
\ldots          &  \verb!\ldots! &
\textbullet     &  \verb!\textbullet! \\
\$              &  \verb!\$! &
\^{}            &  \verb!\^{}! &
\textbar        &  \verb!\textbar! &
\textbackslash  &  \verb!\textbackslash! \\
\%              &  \verb!\%! &
\~{}            &  \verb!\~{}! &
\#              &  \verb!\#! &
\S              &  \verb!\S! \\
\end{tabular}

\subsection{Accents}
\begin{tabular}{@{}l@{\ }l|l@{\ }l|l@{\ }l|l@{\ }l|l@{\ }l@{}}
\`o   & \verb!\`o! &
\'o   & \verb!\'o! &
\^o   & \verb!\^o! &
\~o   & \verb!\~o! &
\=o   & \verb!\=o! \\
\.o   & \verb!\.o! &
\"o   & \verb!\"o! &
\c o  & \verb!\c o! &
\v o  & \verb!\v o! &
\H o  & \verb!\H o! \\
\c c  & \verb!\c c! &
\d o  & \verb!\d o! &
\b o  & \verb!\b o! &
\t oo & \verb!\t oo! &
\oe   & \verb!\oe! \\
\OE   & \verb!\OE! &
\ae   & \verb!\ae! &
\AE   & \verb!\AE! &
\aa   & \verb!\aa! &
\AA   & \verb!\AA! \\
\o    & \verb!\o! &
\O    & \verb!\O! &
\l    & \verb!\l! &
\L    & \verb!\L! &
\i    & \verb!\i! \\
\j    & \verb!\j! &
!`    & \verb!~`! &
?`    & \verb!?`! &
\end{tabular}


\subsection{Delimiters}
\begin{tabular}{@{}l@{\ }ll@{\ }ll@{\ }ll@{\ }ll@{\ }ll@{\ }l@{}}
`       & \verb!`!  &
``      & \verb!``! &
\{      & \verb!\{! &
\lbrack & \verb![! &
(       & \verb!(! &
\textless  &  \verb!\textless! \\
'       & \verb!'!  &
''      & \verb!''! &
\}      & \verb!\}! &
\rbrack & \verb!]! &
)       & \verb!)! &
\textgreater  &  \verb!\textgreater! \\
\end{tabular}

\subsection{Dashes}
\begin{tabular}{@{}llll@{}}
\textit{Name} & \textit{Source} & \textit{Example} & \textit{Usage} \\
hyphen  & \verb!-!   & X-ray          & In words. \\
en-dash & \verb!--!  & 1--5           & Between numbers. \\
em-dash & \verb!---! & Yes---or no?    & Punctuation.
\end{tabular}


\subsection{Line and page breaks}
\settowidth{\MyLen}{\texttt{.pagebreak} }
\begin{tabular}{@{}p{\the\MyLen}%
                @{}p{\linewidth-\the\MyLen}@{}}
\verb!\\!          &  Begin new line without new paragraph.  \\
\verb!\\*!         &  Prohibit pagebreak after linebreak. \\
\verb!\kill!       &  Don't print current line. \\
\verb!\pagebreak!  &  Start new page. \\
\verb!\noindent!   &  Do not indent current line.
\end{tabular}


\subsection{Miscellaneous}
\settowidth{\MyLen}{\texttt{.rule.w..h.} }
\begin{tabular}{@{}p{\the\MyLen}%
                @{}p{\linewidth-\the\MyLen}@{}}
\verb!\today!  &  \today. \\
\verb!$\sim$!  &  Prints $\sim$ instead of \verb!\~{}!, which makes \~{}. \\
\verb!~!       &  Space, disallow linebreak (\verb!W.J.~Clinton!). \\
\verb!\@.!     &  Indicate that the \verb!.! ends a sentence when following
                        an uppercase letter. \\
\verb!\hspace{!$l$\verb!}! & Horizontal space of length $l$
                                (Ex: $l=\mathtt{20pt}$). \\
\verb!\vspace{!$l$\verb!}! & Vertical space of length $l$. \\
\verb!\rule{!$w$\verb!}{!$h$\verb!}! & Line of width $w$ and height $h$. \\
\end{tabular}



\section{Tabular environments}

\subsection{\texttt{tabbing} environment}
\begin{tabular}{@{}l@{\hspace{1.5ex}}l@{\hspace{10ex}}l@{\hspace{1.5ex}}l@{}}
\verb!\=!  &   Set tab stop. &
\verb!\>!  &   Go to tab stop.
\end{tabular}

Tab stops can be set on ``invisible'' lines with \verb!\kill!
at the end of the line.  Normally \verb!\\! is used to separate lines.


\subsection{\texttt{tabular} environment}
\verb!\begin{array}[!\textit{pos}\verb!]{!\textit{cols}\verb!}!   \\
\verb!\begin{tabular}[!\textit{pos}\verb!]{!\textit{cols}\verb!}! \\
\verb!\begin{tabular*}{!\textit{width}\verb!}[!\textit{pos}\verb!]{!\textit{cols}\verb!}!


\subsubsection{\texttt{tabular} column specification}
\settowidth{\MyLen}{\texttt{p}\{\textit{width}\} \ }
\begin{tabular}{@{}p{\the\MyLen}@{}p{\linewidth-\the\MyLen}@{}}
\texttt{l}    &   Left-justified column.  \\
\texttt{c}    &   Centered column.  \\
\texttt{r}    &   Right-justified column. \\
\verb!p{!\textit{width}\verb!}!  &  Same as %
                              \verb!\parbox[t]{!\textit{width}\verb!}!. \\ 
\verb!@{!\textit{decl}\verb!}!   &  Insert \textit{decl} instead of
                                    inter-column space. \\
\verb!|!      &   Inserts a vertical line between columns. 
\end{tabular}


\subsubsection{\texttt{tabular} elements}
\settowidth{\MyLen}{\texttt{.cline.x-y..}}
\begin{tabular}{@{}p{\the\MyLen}@{}p{\linewidth-\the\MyLen}@{}}
\verb!\hline!           &  Horizontal line between rows.  \\
\verb!\cline{!$x$\verb!-!$y$\verb!}!  &
                        Horizontal line across columns $x$ through $y$. \\
\verb!\multicolumn{!\textit{n}\verb!}{!\textit{cols}\verb!}{!\textit{text}\verb!}! \\
        &  A cell that spans \textit{n} columns, with \textit{cols} column specification.
\end{tabular}

\section{Math mode}
For inline math, use \verb!\(...\)! or \verb!$...$!.
For displayed math, use \verb!\[...\]! or \verb!\begin{equation}!.

\begin{tabular}{@{}l@{\hspace{1em}}l@{\hspace{2em}}l@{\hspace{1em}}l@{}}
Superscript$^{x}$       &
\verb!^{x}!             &  
Subscript$_{x}$         &
\verb!_{x}!             \\  
$\frac{x}{y}$           &
\verb!\frac{x}{y}!      &  
$\sum_{k=1}^n$          &
\verb!\sum_{k=1}^n!     \\  
$\sqrt[n]{x}$           &
\verb!\sqrt[n]{x}!      &  
$\prod_{k=1}^n$         &
\verb!\prod_{k=1}^n!    \\ 
\end{tabular}

\subsection{Math-mode symbols}

% The ordering of these symbols is slightly odd.  This is because I had to put all the
% long pieces of text in the same column (the right) for it all to fit properly.
% Otherwise, it wouldn't be possible to fit four columns of symbols here.

\begin{tabular}{@{}l@{\hspace{1ex}}l@{\hspace{1em}}l@{\hspace{1ex}}l@{\hspace{1em}}l@{\hspace{1ex}} l@{\hspace{1em}}l@{\hspace{1ex}}l@{}}
$\leq$          &  \verb!\leq!  &
$\geq$          &  \verb!\geq!  &
$\neq$          &  \verb!\neq!  &
$\approx$       &  \verb!\approx!  \\
$\times$        &  \verb!\times!  &
$\div$          &  \verb!\div!  &
$\pm$           & \verb!\pm!  &
$\cdot$         &  \verb!\cdot!  \\
$^{\circ}$      & \verb!^{\circ}! &
$\circ$         &  \verb!\circ!  &
$\prime$        & \verb!\prime!  &
$\cdots$        &  \verb!\cdots!  \\
$\infty$        & \verb!\infty!  &
$\neg$          & \verb!\neg!  &
$\wedge$        & \verb!\wedge!  &
$\vee$          & \verb!\vee!  \\
$\supset$       & \verb!\supset!  &
$\forall$       & \verb!\forall!  &
$\in$           & \verb!\in!  &
$\rightarrow$   &  \verb!\rightarrow! \\
$\subset$       & \verb!\subset!  &
$\exists$       & \verb!\exists!  &
$\notin$        & \verb!\notin!  &
$\Rightarrow$   &  \verb!\Rightarrow! \\
$\cup$          & \verb!\cup!  &
$\cap$          & \verb!\cap!  &
$\mid$          & \verb!\mid!  &
$\Leftrightarrow$   &  \verb!\Leftrightarrow! \\
$\dot a$        & \verb!\dot a!  &
$\hat a$        & \verb!\hat a!  &
$\bar a$        & \verb!\bar a!  &
$\tilde a$      & \verb!\tilde a!  \\

$\alpha$        &  \verb!\alpha!  &
$\beta$         &  \verb!\beta!  &
$\gamma$        &  \verb!\gamma!  &
$\delta$        &  \verb!\delta!  \\
$\epsilon$      &  \verb!\epsilon!  &
$\zeta$         &  \verb!\zeta!  &
$\eta$          &  \verb!\eta!  &
$\varepsilon$   &  \verb!\varepsilon!  \\
$\theta$        &  \verb!\theta!  &
$\iota$         &  \verb!\iota!  &
$\kappa$        &  \verb!\kappa!  &
$\vartheta$     &  \verb!\vartheta!  \\
$\lambda$       &  \verb!\lambda!  &
$\mu$           &  \verb!\mu!  &
$\nu$           &  \verb!\nu!  &
$\xi$           &  \verb!\xi!  \\
$\pi$           &  \verb!\pi!  &
$\rho$          &  \verb!\rho!  &
$\sigma$        &  \verb!\sigma!  &
$\tau$          &  \verb!\tau!  \\
$\upsilon$      &  \verb!\upsilon!  &
$\phi$          &  \verb!\phi!  &
$\chi$          &  \verb!\chi!  &
$\psi$          &  \verb!\psi!  \\
$\omega$        &  \verb!\omega!  &
$\Gamma$        &  \verb!\Gamma!  &
$\Delta$        &  \verb!\Delta!  &
$\Theta$        &  \verb!\Theta!  \\
$\Lambda$       &  \verb!\Lambda!  &
$\Xi$           &  \verb!\Xi!  &
$\Pi$           &  \verb!\Pi!  &
$\Sigma$        &  \verb!\Sigma!  \\
$\Upsilon$      &  \verb!\Upsilon!  &
$\Phi$          &  \verb!\Phi!  &
$\Psi$          &  \verb!\Psi!  &
$\Omega$        &  \verb!\Omega!  
\end{tabular}
\footnotesize

%\subsection{Special symbols}
%\begin{tabular}{@{}ll@{}}
%$^{\circ}$  &  \verb!^{\circ}! Ex: $22^{\circ}\mathrm{C}$: \verb!$22^{\circ}\mathrm{C}$!.
%\end{tabular}

\section{Bibliography and citations}
When using \BibTeX, you need to run \texttt{latex}, \texttt{bibtex},
and \texttt{latex} twice more to resolve dependencies.

\subsection{Citation types}
\settowidth{\MyLen}{\texttt{.shortciteN.key..}}
\begin{tabular}{@{}p{\the\MyLen}@{}p{\linewidth-\the\MyLen}@{}}
\verb!\cite{!\textit{key}\verb!}!       &
        Full author list and year. (Watson and Crick 1953) \\
\verb!\citeA{!\textit{key}\verb!}!      &
        Full author list. (Watson and Crick) \\
\verb!\citeN{!\textit{key}\verb!}!      &
        Full author list and year. Watson and Crick (1953) \\
\verb!\shortcite{!\textit{key}\verb!}!  &
        Abbreviated author list and year. ? \\
\verb!\shortciteA{!\textit{key}\verb!}! &
        Abbreviated author list. ? \\
\verb!\shortciteN{!\textit{key}\verb!}! &
        Abbreviated author list and year. ? \\
\verb!\citeyear{!\textit{key}\verb!}!   &
        Cite year only. (1953) \\
\end{tabular}

All the above have an \texttt{NP} variant without parentheses;
Ex. \verb!\citeNP!.


\subsection{\BibTeX\ entry types}
\settowidth{\MyLen}{\texttt{.mastersthesis.}}
\begin{tabular}{@{}p{\the\MyLen}@{}p{\linewidth-\the\MyLen}@{}}
\verb!@article!         &  Journal or magazine article. \\
\verb!@book!            &  Book with publisher. \\
\verb!@booklet!         &  Book without publisher. \\
\verb!@conference!      &  Article in conference proceedings. \\
\verb!@inbook!          &  A part of a book and/or range of pages. \\
\verb!@incollection!    &  A part of book with its own title. \\
%\verb!@manual!          &  Technical documentation. \\
%\verb!@mastersthesis!   &  Master's thesis. \\
\verb!@misc!            &  If nothing else fits. \\
\verb!@phdthesis!       &  PhD. thesis. \\
\verb!@proceedings!     &  Proceedings of a conference. \\
\verb!@techreport!      &  Tech report, usually numbered in series. \\
\verb!@unpublished!     &  Unpublished. \\
\end{tabular}

\subsection{\BibTeX\ fields}
\settowidth{\MyLen}{\texttt{organization.}}
\begin{tabular}{@{}p{\the\MyLen}@{}p{\linewidth-\the\MyLen}@{}}
\verb!address!         &  Address of publisher.  Not necessary for major
                                publishers.  \\
\verb!author!           &  Names of authors, of format .... \\
\verb!booktitle!        &  Title of book when part of it is cited. \\
\verb!chapter!          &  Chapter or section number. \\
\verb!edition!          &  Edition of a book. \\
\verb!editor!           &  Names of editors. \\
\verb!institution!      &  Sponsoring institution of tech.\ report. \\
\verb!journal!          &  Journal name. \\
\verb!key!              &  Used for cross ref.\ when no author. \\
\verb!month!            &  Month published. Use 3-letter abbreviation. \\
\verb!note!             &  Any additional information. \\
\verb!number!           &  Number of journal or magazine. \\
\verb!organization!     &  Organization that sponsors a conference. \\
\verb!pages!            &  Page range (\verb!2,6,9--12!). \\
\verb!publisher!        &  Publisher's name. \\
\verb!school!           &  Name of school (for thesis). \\
\verb!series!           &  Name of series of books. \\
\verb!title!            &  Title of work. \\
\verb!type!             &  Type of tech.\ report, ex. ``Research Note''. \\
\verb!volume!           &  Volume of a journal or book. \\
\verb!year!             &  Year of publication. \\
\end{tabular}
Not all fields need to be filled.  See example below.

\subsection{Common \BibTeX\ style files}
\begin{tabular}{@{}l@{\hspace{1em}}l@{\hspace{3em}}l@{\hspace{1em}}l@{}}
\verb!abbrv!    &  Standard &
\verb!abstract! &  \texttt{alpha} with abstract \\
\verb!alpha!    &  Standard &
\verb!apa!      &  APA \\
\verb!plain!    &  Standard &
\verb!unsrt!    &  Unsorted \\
\end{tabular}

The \LaTeX\ document should have the following two lines just before
\verb!\end{document}!, where \verb!bibfile.bib! is the name of the
\BibTeX\ file.
\begin{verbatim}
\bibliographystyle{plain}
\bibliography{bibfile}
\end{verbatim}

\subsection{\BibTeX\ example}
The \BibTeX\ database goes in a file called
\textit{file}\texttt{.bib}, which is processed with \verb!bibtex file!. 
\begin{verbatim}
  author  = {James Watson and Francis Crick},
  title   = {A structure for Deoxyribose Nucleic Acid},
  journal = N,
  volume  = {171},
  pages   = {737},
  year    = 1953
}
\end{verbatim}


\section{Sample \LaTeX\ document}
\begin{verbatim}
\documentclass[11pt]{article}
\usepackage{fullpage}
\title{Template}
\author{Name}
\begin{document}
\maketitle

\section{section}
\subsection*{subsection without number}
text \textbf{bold text} text. Some math: $2+2=5$
\subsection{subsection}
text \emph{emphasized text} text. \cite{WC:1953}
discovered the structure of DNA.

A table:
\begin{table}[!th]
\begin{tabular}{|l|c|r|}
\hline
first  &  row  &  data \\
second &  row  &  data \\
\hline
\end{tabular}
\caption{This is the caption}
\label{ex:table}
\end{table}

The table is numbered \ref{ex:table}.
\end{document}
\end{verbatim}



\rule{0.3\linewidth}{0.25pt}
\scriptsize

Copyright \copyright\ 2012 Winston Chang

http://www.stdout.org/$\sim$winston/latex/


\end{multicols}
\end{document}
